%\newglossaryentry{potato}{name={potato}, description={starchy tuber}}
%\newglossaryentry{cabbage}{name={cabbage}, description={vegetable with thick green or purple leaves}}
%\newglossaryentry{carrot}{name={carrot}, description={orange root}}
%\newglossaryentry{pva}{name={ПВК}, description={Пировиноградная кислота}}
\newacronym{pva}{ПВК}{Пировиноградная кислота}
\newacronym{fgal}{ФГА}{3-Фосфоглицериновый альдегид-1,3}
\newacronym{fgac}{ФГК}{Фосфоглицериновая кислота}
%\newacronym{fgac}{ФГК}{Фосфоглицериновая кислота}
\newacronym{phtr}{ФАР}{Фотосинтетически активная радиация}

\newacronym{atp}{АТФ}{Аденозин три фосфат}

\newacronym[description={кофермент, имеющийся во всех живых клетках}]{nadh}{НАД}{Никотинамидадениндинуклеотид}

\newacronym{fs1}{ФС1}{Фотосистема 1}

\newacronym{fs2}{ФС2}{Фотосистема 2}

\newacronym{electonykLink}{ЭТЦ}{Электрон-транспортная цепь}

\newacronym[description={эфир фосфорной кислоты и енольной формы пировиноградной кислоты}]{fep}{ФЕП}{фосфоенолпируват}

\newacronym[description={широко распространённый в природе кофермент некоторых дегидрогеназ}]{nadfh2}{НАДФН2}{Никотинамидадениндинуклеотидфосфат}

\newacronym[description={сложное органическое вещество, молекулы которого участвуют в главнейших биохимических реакциях, идущих в живой клетке}]{acetylCoensimA}{Ацетил-КоА}{комплекс ацетильной группы и СоA}

\newacronym{kpd}{КПД}{Коэффициент полезного действия}

\newacronym[description={единица измерения молекулярных масс высокомолекулярных соединений, например белков, углеводов, гуминовых кислот. 1 Да = 1 г/моль}]{dalton}{Да}{Дальтон}

\newacronym[description={единица измерения давления. Паскаль равен давлению, вызываемому силой, равной одному ньютону, равномерно распределённой по перпендикулярной к ней поверхности площадью один квадратный метр}]{pascal}{Па}{Паскаль}

\newacronym[description={количество вещества системы, содержащей столько же структурных элементов, сколько содержится атомов в углероде-12 массой 0,012 кг}]{mol}{М}{Моль}

\newacronym[see={[см. также]{auxin}}]{inodolAcid}{ИУК}{Инодол уксусная кислота}

\newacronym{abscizeAcid}{АБК}{Абсцизовая кислота}

\newacronym{rna}{РНК}{Рибонуклеиновая кислота}

\newacronym{dna}{ДНК}{Дезоксирибонуклеиновая кислота}

\newacronym[description={класс белков, экспрессия которых усиливается при повышении температуры или при других стрессирующих клетку условиях. Повышение экспрессии генов, кодирующих белки теплового шока, регулируется на этапе транскрипции}]{HeatShockProteins}{БТШ}{Белки теплового шока}

\newacronym{frostShockProteins}{БХШ}{Белки холодового шока}

\newglossaryentry{SoilQquantOfWater}
{
%type=notation,
name={Влагоемкость почвы},
description={величина, количественно характеризующая водоудерживающую способность почвы}
}

\newglossaryentry{nutrientElements}
{
%type=notation,
name={Питательные элементы},
description={химические элементы, которые необходимы растению и не могут быть заменены никакими другими. }
}

\newglossaryentry{nutrients}
{
%type=notation,
name={Питательные вещества},
description={соединения, в которых имеются питательные элементы}
}

\newglossaryentry{photosyntesis}
{
%type=notation,
name={Фотосинтез},
description={процесс синтеза органических веществ из неорганических при участии хлорофилла и за счет энергии солнца.}
}

\newglossaryentry{endocytosis}
{
%type=notation,
name={Эндоцитоз},
description={активный способ поглощение макромолекул клеткой, который сопровождается впячиванием мембраны и образованием везикул, внутри которых содержатся макромолекулы}
}

\newglossaryentry{activeTranspotr}
{
%type=notation,
name={Активный транспорт веществ},
description={транспорт веществ, идущий через цитоплазматическую мембрану против электрохимического потенциала с затратой энергии, выделяющейся в процессе метаболизма в форме АТФ}
}

\newglossaryentry{passiveTranspotr}
{
%type=notation,
name={Пассивный транспорт веществ},
description={транспорт веществ через цитоплазматическую мембрану, идущий без затраты энергии, по градиенту электрохимического потенциала}
}

\newglossaryentry{avtotrophicOrg}
{
%type=notation,
name={Автотрофные организмы},
description={организмы, способные синтезировать органические вещества из неорганических путем фото- или хемосинтеза}
}

\newglossaryentry{photosystem}
{
%type=notation,
name={Фотосистема},
description={белковый комплекс, который осуществляет первичные фотохимические реакции фотосинтеза: поглощение света, преобразование энергии и перенос электронов}
}

\newglossaryentry{fzrsc}
{
%type=notation,
name={Физиология растений},
description={наука, изучающая общие закономерности жизнедеятельности растительных организмов и является частью биологической науки}
}

\newglossaryentry{ammonification}
{
%type=notation,
name={Аммонификация},
description={процесс превращения органического азота почвы в $NH^{+}_{4}$-ионы}
}

\newglossaryentry{nitrification}
{
%type=notation,
name={Нитрификация},
description={процесс биологического окисления аммония $NH^{+}_{4}$ до нитрат-ионов $NO^{-}_{3}$}
}

\newglossaryentry{nitrogenaza}
{
%type=notation,
name={Нитрогеназа},
description={комплекс ферментов (мультифермент), осуществляющий процесс фиксации атмосферного азота. Широко распространён у бактерий и архей, в то время как все эукариоты его лишены}
}

\newglossaryentry{leggemoglobin}
{
%type=notation,
name={Леггемоглобин},
description={разновидность гемоглобина, содержащаяся в клубеньках бобовых растений и придающая им красный цвет. Леггемоглобин способствует переносу кислорода в симбиосомы, содержащие азотфиксирующие бактерии, для их дыхания.}
}

\newglossaryentry{growth}
{
%type=notation,
name={Рост},
description={необратимое увеличение размеров и массы клетки, органа или всего организма растения, связанное с новообразованием элементов составляющих его структуру}
}

\newglossaryentry{evolution}
{
%type=notation,
name={Развитие},
description={качественные изменения в структуре и функциональной активности растения и его частей в процессе онтогенеза}
}

\newglossaryentry{ontogenesis}
{
%type=notation,
name={Онтогенез},
description={индивидуальное развитие организма от зиготы или вегетативного зачатка до естественной смерти}
}

\newglossaryentry{diferintation}
{
%type=notation,
name={Дифференцировка},
description={индивидуальное развитие организма от зиготы или вегетативного зачатка до естественной смерти}
}

\newglossaryentry{mitosis}
{
%type=notation,
name={Митоз},
description={такой способ деления клеток, при котором число хромосом удваивается, так что каждая дочерняя клетка получает набор хромосом, равный набору хромосом материнской клетки.}
}

\newglossaryentry{genExpression}
{
%type=notation,
name={Экспрессия генов},
description={изменение активности генов}
}

\newglossaryentry{totipatentia}
{
%type=notation,
name={Тотипатентность},
description={способность клетки дать начало всему организму}
}

\newglossaryentry{tonoplast}
{
%type=notation,
name={Тонопласт},
description={мембранна центральной вакуоли растительной клетки}
}

\newglossaryentry{monokarpics}
{
%type=notation,
name={Монокарпические растения},
description={растения, плодоносящие один раз в жизни}
}

\newglossaryentry{polykarpics}
{
%type=notation,
name={Поликарпические растения},
description={растения, многократно плодоносящие в течении жизни}
}

\newglossaryentry{generativeOrgans}
{
%type=notation,
name={Генеративные органы},
description={органы полового размножения растений}
}

\newglossaryentry{partenokarpia}
{
%type=notation,
name={Партенокарпия},
description={процесс образования плодов без оплодотворения и образования семян}
}

\newglossaryentry{meristema}
{
%type=notation,
name={Меристемы},
description={группа образовательных растительных тканей. Клетки меристемы постоянно делятся}
}

\newglossaryentry{morphogenesis}
{
%type=notation,
name={Морфогенез},
description={генетически запрограммированный процесс образования клеток, тканей, органов}
}

\newglossaryentry{polarisation}
{
%type=notation,
name={Поляризация},
description={ориентация процессов и структур в пространстве, то есть физиолого-биохимические и анатомо-морфологические свойства изменяются в определенном направлении}
}

\newglossaryentry{growthCorrelations}
{
%type=notation,
name={Ростовые корреляции},
description={зависимость роста и развития одних органов от других},
see={growth}
}

\newglossaryentry{growthRate}
{
%type=notation,
name={Удельная скорость роста},
description={прирост массы растения или отдельного его органа в единицу времени},
see={growth}
}

\newglossaryentry{relativeGrowthRate}
{
%type=notation,
name={Относительный или процентный рост},
description={прирост, вычисленный в процентах от исходного веса растения или органа},
see={growth}
}
 
\newglossaryentry{absoluteGrowthRate}
{
%type=notation,
name={Абсолютная скорость роста},
description={величина прироста за промежуток времени, отнесенная к единице времени},
see={growth}
}

\newglossaryentry{necessityRepose}
{
%type=notation,
name={Вынужденный покой},
description={покой растения, вызванный воздействием факторов внешней среды}
}

\newglossaryentry{physiologycalRepose}
{
%type=notation,
name={Физиологический покой},
description={покой растения, который регулируется балансом стимуляторов и ингибиторов роста}
}

\newglossaryentry{stratification}
{
%type=notation,
name={Стратификация},
description={процесс выдерживания влажных семян при пониженной температуре}
}

\newglossaryentry{retardants}
{
%type=notation,
name={Ретарданты},
description={синтетические регуляторы роста, которые подавляют рост стебля благодаря торможению растяжения клеток и подавлению синтеза гиббереллинов}
}

\newglossaryentry{morphactins}
{
%type=notation,
name={Морфактины},
description={синтетические регуляторы роста, которые препятствуют прорастанию семян, образованию и росту побегов, ослабляют апикальное доминирование у побегов и усиливают его у корней.}
}

\newglossaryentry{depholiants}
{
%type=notation,
name={Дефолианты},
description={синтетические регуляторы роста, которые ускоряют листопад у растений, что активирует созревание семян и плодов и облегчает механизированную уборку урожая}
}

\newglossaryentry{denaturation}
{
%type=notation,
name={Денатурация},
description={изменение нативной конформации белковой молекулы под действием различных дестабилизирующих факторов. Аминокислотная последовательность белка не изменяется}
}

\newglossaryentry{diffusion}
{
%type=notation,
name={Диффузия},
description={изменение нативной конформации белковой молекулы под действием различных дестабилизирующих факторов. Аминокислотная последовательность белка не изменяется}
}

\newglossaryentry{vitamines}
{
%type=notation,
name={Витамины},
description={низкомолекулярные физиологически активные органические соединения различного химического состава}
}

\newglossaryentry{stress}
{
%type=notation,
name={Стресс},
description={общая неспецифическая адаптационная реакция организма на действие любых неблагоприятных факторов}
}

\newglossaryentry{stressors}
{
%type=notation,
name={Стрессоры},
description={неблагоприятные факторы внешней среды}
}

\newglossaryentry{coldResistance}
{
%type=notation,
name={Холодоустойчивость},
description={способность растений переносить действие положительных температур, близких к нулю градусов}
}

\newglossaryentry{frostResistance}
{
%type=notation,
name={Морозостойкость},
description={способность растений переносить температуру ниже нуля градусов}
}

\newglossaryentry{gassResistance}
{
%type=notation,
name={Газоустойчивость},
description={способность растений сохранять жизнедеятельность при действии вредных газов}
}

\newglossaryentry{crioprotectors}
{
%type=notation,
name={Криопротекторы},
description={вещества, защищающие цитоплазму клетки от образования в ней кристаллов льда}
}

\newglossaryentry{gallophites}
{
%type=notation,
name={Галлофиты},
description={растения, устойчивые к засолению почв}
}

\newglossaryentry{glycophites}
{
%type=notation,
name={Гликофиты},
description={растения, незасоленых водоемов и почв}
}

\newglossaryentry{sexualMaturing}
{
%type=notation,
name={Половое размножение},
description={тип размножения, связанный с образованием и слиянием специализированных половых клеток -- гамет}
}

\newglossaryentry{asexualMaturing}
{
%type=notation,
name={Бесполое размножение},
description={тип размножения, когда новый организм появляется из спор}
}

\newglossaryentry{vegetationMaturing}
{
%type=notation,
name={Вегетативное размножение},
description={воспроизведение растений из вегетативных частей растения (клубней, луковиц, отводок)}
}

\newglossaryentry{turgor}
{
%type=notation,
name={Тургорное давление},
description={внутреннее давление, которое развивается в растительной клетке, когда в неё в результате осмоса входит вода, и цитоплазма прижимается к клеточной стенке}
}

\newglossaryentry{vesicula}
{
%type=notation,
name={Везикула},
description={относительно маленькие внутриклеточные органоиды, мембрано-защищённые сумки, в которых запасаются или транспортируются питательные вещества}
}

\newglossaryentry{breazingIndex}
{
%type=notation,
name={Дыхательный коэффициент},
description={отношение объема выделившегося углекислого газа к объему поглощенного кислорода}
}

\newglossaryentry{VanGoffRule}
{
%type=notation,
name={Правило Ван-Гоффа},
description={при изменении температуры на 10~\celsius скорость реакции изменяется в 2-4 раза}
}

\newglossaryentry{transpiration}
{
%type=notation,
name={Транспирация},
description={физиологически активное испарение воды растением}
}

\newglossaryentry{porins}
{
%type=notation,
name={Порины},
description={трансмембранные белки, представляющие собой гидрофильные поры в липофильной мембране}
}

\newglossaryentry{protonsPomp}
{
%type=notation,
name={Протонная помпа},
description={интегральный мембранный белок, осуществляющий перемещение протонов через мембрану. Термин «помпа» показывает, что поступление идет с потреблением свободной энергии и против электрохимического градиента}
}

\newglossaryentry{phosphriling}
{
%type=notation,
name={Фосфорилирование},
description={процесс переноса остатка фосфорной кислоты от донора к субстрату, как правило, катализируемый ферментами и ведущий к образованию сложных эфиров фосфорной кислоты}
}

\newglossaryentry{nitritreductaza}
{
%type=notation,
name={Нитритредуктаза},
description={сложный фермент, катализирующий восстановление нитрита до аммиака в процессе ассимиляции нитрата}
}

\newglossaryentry{cellCycle}
{
%type=notation,
name={Клеточный цикл},
description={период существования клетки от момента её образования путём деления материнской клетки до собственного деления или гибели}
}

\newglossaryentry{bacteroids}
{
%type=notation,
name={Бактероиды},
description={форма клубеньковой бактерии (род Rhizobium), образующаяся после проникновения в корни бобовых растений, имеет более крупные размеры клеток, высокое содержание жира, гликогена и др.}
}

\newglossaryentry{chlorophill}
{
%type=notation,
name={Хлорофилл},
description={зелёный пигмент, окрашивающий хлоропласты растений в зелёный цвет. При его участии осуществляется процесс фотосинтеза. По химическому строению хлорофиллы — магниевые комплексы различных тетрапирролов}
}

\newglossaryentry{metabolism}
{
%type=notation,
name={Метаболизм},
description={набор химических реакций, которые возникают в живом организме для поддержания жизни. Эти процессы позволяют организмам расти и размножаться, сохранять свои структуры и отвечать на воздействия окружающей среды}
}

\newglossaryentry{plastocyanin}
{
%type=notation,
name={Пластоцианин},
description={небольшой водорастворимый белок, основная функция которого заключается в переносе электронов от цитохромного bf комплекса к фотосистеме 1},
see={cytochrom}
}


\newglossaryentry{glycolisys}
{
%type=notation,
name={Гликолиз},
description={процесс анаэробного окисления глюкозы, при котором из одной молекулы глюкозы образуются две молекулы пировиноградной кислоты}
}

\newglossaryentry{auxin}
{
%type=notation,
name={Ауксин},
description={природный регулятор роста растений (фитогормонов). Влияет на рост, деление и дифференциацию клеток; играет важную роль в явлениях гео- и фототропизма},
see={phitogormons}
}

\newglossaryentry{coenzim}
{
%type=notation,
name={Кофермент},
description={малая молекула небелковой природы, специфически соединяющаяся с соответствующими белками, называемыми апоферментами, и играющая роль активного центра или простетической группы молекулы фермента}
}

\newglossaryentry{proton}
{
%type=notation,
name={Протон},
description={элементарная частица обладающая положительным зарядом. В данном пособии под словом <<протон>> подразумевается ион водорода H+}
}

\newglossaryentry{jarovisation}
{
%type=notation,
name={Яровизация},
description={побуждение семян или растений к росту и более интенсивному развитию с помощью непродолжительного воздействия низких положительных температур}
}

\newglossaryentry{photoperiodism}
{
%type=notation,
name={Фотопериодизм},
description={реакция живых организмов (растений и животных) на суточный ритм освещённости, продолжительность светового дня и соотношение между темным и светлым временем суток}
}

\newglossaryentry{phitochrom}
{
%type=notation,
name={Фитохром},
description={фоторецептор, сине-зеленый пигмент, существующий в двух взаимопревращающихся формах. Поглотив свет, фитохром переходит из одной формы в другую. Этот пигмент играет важную роль в ряде процессов, таких как цветение и прорастание семян}
}

\newglossaryentry{initialCells}
{
%type=notation,
name={Инициалии},
description={клетки меристем, способные неопределенно долго делиться в результате митоза},
see={meristema}
}

\newglossaryentry{apexDominance}
{
%type=notation,
name={Апикальное доминирование},
description={торможение верхушечной почкой побега развития боковых побегов из пазушных почек},
see={apex}
}

\newglossaryentry{phitogormons}
{
%type=notation,
name={Фитогормоны},
description={низкомолекулярные органические вещества, вырабатываемые растениями и имеющие регуляторные функции}
%see={meristema}
}

\newglossaryentry{gibberelins}
{
%type=notation,
name={Гибберелины},
description={группа фитогормонов дитерпеновой природы, которые выполняют в растениях разнообразные функции, связанные с контролем удлинения гипокотиля, прорастания семян, зацветания и т. д.},
see={phitogormons}
}

\newglossaryentry{apex}
{
%type=notation,
name={Апекс},
description={верхушка побега или корня, представленная первичной меристемой; обеспечивает верхушечный, или апикальный, рост этих органов: образование новых метамеров побега и удлинение корня}
}

\newglossaryentry{ephemers}
{
%type=notation,
name={Эфимеры},
description={экологическая группа травянистых однолетних растений с очень коротким вегетационным периодом (некоторые заканчивают полный цикл своего развития всего за несколько недель)},
see={life_cycle}
}

\newglossaryentry{life_cycle}
{
%type=notation,
name={Жизненный цикл},
description={закономерная смена всех поколений, характерных для данного вида живых организмов}
}

\newglossaryentry{sporophit}
{
%type=notation,
name={Спорофит},
description={диплоидная многоклеточная фаза в жизненном цикле растений и водорослей, развивающаяся из оплодотворенной яйцеклетки или зиготы и производящая споры},
see={life_cycle}
}

\newglossaryentry{gametophit}
{
%type=notation,
name={Гаметофит},
description={гаплоидная многоклеточная фаза в жизненном цикле растений и водорослей, развивающаяся из спор и производящая половые клетки},
see={life_cycle}
}

\newglossaryentry{suctionPressure}
{
%type=notation,
name={Гаметофит},
description={величина, равная разности осмотического и тургорного давления},
see={turgor}
}

\newglossaryentry{termoresistens}
{
%type=notation,
name={Термоустойчивость},
description={адаптивная устойчивость белков, клеток, органов и целых организмов к экстремальным положительным температурам}
}

\newglossaryentry{gipoxia}
{
%type=notation,
name={Гипоксия},
description={патологической состояние, характеризующееся дефицитом кислорода в организме}
}

\newglossaryentry{pneumatophores}
{
%type=notation,
name={Пневматофоры},
description={надземные дыхательные корни растений, растущие вверх}
}

\newglossaryentry{aerenhima}
{
%type=notation,
name={Аэренхима},
description={воздухоносная ткань у растений, построенная из клеток, соединённых между собой так, что между ними остаются крупные заполненные воздухом межклетники}
}

\newglossaryentry{reparation}
{
%type=notation,
name={Репарация},
description={особая функция клеток, заключающаяся в способности исправлять химические повреждения и разрывы в молекулах ДНК, повреждённой при нормальном биосинтезе ДНК в клетке или в результате воздействия физических или химических агентов}
}

\newglossaryentry{radioprotectors}
{
%type=notation,
name={Радиопротекторы},
description={вещества, повышающие устойчивость организма к воздействию ионизирующих излучений}
}

\newglossaryentry{fitoncides}
{
%type=notation,
name={Фитонциды},
description={выделяемые растениями биологически активные вещества, убивающие или подавляющие рост и развитие болезнетворных бактерий}
}

\newglossaryentry{patogen}
{
%type=notation,
name={Патоген},
description={любой микроорганизм (включая грибы, вирусы, бактерии, и проч.), а также особый белок — прион, способный вызывать болезнь другого живого существа}
}

\newglossaryentry{anoxicOxigenation}
{
%type=notation,
name={Аноксическое окисление},
description={окисление, в ходе которого электроны передаются не на кислород а на другие вещества -- нитраты и двойные связи ненасыщенных соединений}
}

\newglossaryentry{biotroph}
{
%type=notation,
name={Биотроф},
description={организм, питающийся биомассой других организмов. Биотрофами являются хищники, паразиты и симбионты}
}

\newglossaryentry{saprotroph}
{
%type=notation,
name={Сапротроф},
description={организм, питающийся остатками мертвых организмов},
see={necrotroph}
}

\newglossaryentry{necrotroph}
{
%type=notation,
name={Некротроф},
description={факультативный паразит (и некоторые факультативные сапрофиты), поселяющийся на предварительно убитой им ткани},
see={saprotroph}
}

\newglossaryentry{exoenzims}
{
%type=notation,
name={Экзоферменты},
description={ферменты, не связанные с цитоплазмой клетки, они свободно выделяются во внешнюю среду или субстрат}
}

\newglossaryentry{tiles}
{
%type=notation,
name={Тилы},
description={пузыревидные выросты клеток осевой или лучевой паренхимы, проникающие через поры в стенках сосудов в просветы последних}
}

\newglossaryentry{cytochrom}
{
%type=notation,
name={Цитохром},
description={крупные мембранные белки (за исключением наиболее распространённого цитохрома c, который является маленьким глобулярным белком), которые содержат ковалентно связанный гем, расположенный во внутреннем кармане, образованном аминокислотными остатками. Цитохромы катализируют окислительные реакции}
}

 