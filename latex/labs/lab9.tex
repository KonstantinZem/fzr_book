%\section*{\lbtitle Влияние отдельных элементов питательной смеси на рост растения}
%\addcontentsline{toc}{section}{Влияние отдельных элементов питательной смеси на рост растения}

%\subsection*{Теоретические положения}

%\paragraph*{}Исключение из минерального питания растения хотя бы одного химического элемента приводит к нарушению обмена веществ растительного организма, проявляется в замедление роста растения и приводит к его последующей гибели. При этом наиболее быстро к видимым нарушениям обмена веществ приводит отсутствие таких элементов как азот и кальций. Роль наиболее важных элементов кратко описывается ниже:

%\begin{itemize}

%\item \hyperlink{nitrogen}{Азот} необходим для роста растений, так как входит в состав многих органических веществ, главным образом белков. 

%\item Калий увеличивает водоудерживающую способность протоплазмы клетки, и способствует поддержанию тургорного давления.

%\item \hyperlink{PS}{Сера} входит в состав аминокислот, являющихся мономерами белков. Соединения серы играют большую роль как регуляторы окислительно-восстановительного потенциала живой клетки.

%\item \hyperlink{PS}{Фосфор} входит в состав белков, нуклеиновых кислот, фосфатидов, ферментов, витаминов, фитина и других биологически активных веществ

%\item \hyperlink{MnMg}{Магний} входит в состав хлорофилла и, следовательно, необходим для фотосинтеза. 

%\item Кальций участвует в нейтрализации образующихся в тканях в процессе обмена веществ органических кислот, в частности щавелевой кислоты.

%\end{itemize}

%\paragraph*{}Кроме кальция нереутилизируемым являются многие минеральные элементы.

\begin{footnotesize}

\paragraph*{}\textbf{Цель работы}: Определить \hypertarget{mineral_elements_influence}{влияние недостатка отдельных минеральных элементов} на рост и развитие растений;

\paragraph*{}\textbf{Оборудование}: Стеклянные банки емкостью 1 литр, бумага, деревянные пробки, бюретки на 50 мл, проростки растений;

\paragraph*{}\textbf{Реактивы}: Растворы химически чистых солей - KNO${_3}$, Ca(NO${_3}$)${_2}$, NaCl, KH${_2}$PO${_4}$, NaH${_2}$PO${_4}$, MgSO${_4}$ \textperiodcentered 7H${_2}$O, CaSO${_4}$ \textperiodcentered 2H${_2}O$, MnSO${_4}$, 0,5\% раствор цитрата железа, раствор борной кислоты;

\end{footnotesize}


\subsection*{Ход работы}	
	
\subsubsection*{Приготовление питательной среды}
	
\paragraph*{}Приготовьте для опыта полную питательную смесь по Хогланду-Снайдерсу, а так же смеси из состава которых исключены по отдельности азот, фосфор и калий. Следует учесть, что при исключении из смеси отдельного элемента питания, связанные с ним элементы необходимо внести в эквивалентных количествах, в виде солей, не содержащих исключаемый элемент \footnote{Это делается для чистоты эксперимента, чтобы исключить влияние на растение  связанных элементов. Например удаляя калий в форме   K{H$_2$}PO{$_4$} вместе с ним мы удаляем еще и фосфор входящий в состав иона {H$_2$}PO{$_4^-$}}. 

\paragraph*{}Перед приготовлением питательного раствора составьте рабочую таблицу, где укажите необходимое количество солей на выбранный объем (таблица \ref{hogland_mixture}).

\label{hogland_mixture}	
%\begin{longtable}{|p{0.2\linewidth}|p{0.2\linewidth}|p{0.15\linewidth}|p{0.15\linewidth}|p{0.15\linewidth}|}
\begin{longtable}{|c|c|c|c|c|}
\caption{Рабочая таблица для приготовления питательной смеси}\\
\hline  \multirow{1}{*}{Соль} & \multirow{1}{*}{Масса соли для маточного раствора} & \multicolumn{3}{c|}{Количество маточного раствора} \\ \cline{3-5}                                                                                         
	          
  & & 1 норма                & 0.5 нормы	        & 0.2 нормы	\\

\hline \multicolumn{5}{|c|}{Микроэлементы на 10 л раствора} \\

\hline KNO$_3$ & 510 & 10.0 & 5.0 & 2.0 \\                                                                                  
\hline Ca(NO${_3}$)${_2}$ & 10 \% раствор & 8.2 & 4.1 & 1.6 \\                                                                                  
\hline K{H$_2$}PO{$_4$} & 136 & 10.0 & 5.0 & 2.0 \\                                                                                  
\hline MgSO${_4}$ \textperiodcentered 7H${_2}$O & 490 & 10.0 & 5.0 & 2.0 \\                                                                                  

\hline \multicolumn{5}{|c|}{Микроэлементы на 2 л раствора} \\

\hline MnCl$_{2}$ \textperiodcentered 4H${_2}$O & 0.35 &  &  &  \\                                                                                  
\hline H$_{3}$BO$_{3}$ & 0.55 &  &  &  \\                                                                                  
\hline ZnSO${_4}$ & 0.05 &  &  &  \\
\hline CuSO${_4}$ & 0.05 &  &  &  \\                                                                                  
\hline MoO${_2}$ & 0.024 &  &  &  \\
\hline FeSO${_4}$ \textperiodcentered 7H${_2}$O & 4.0 &  &  &  \\
\hline

\end{longtable}

\paragraph*{\warningsign}Во избежании размножения водорослей, приготовленный раствор нужно хранить в посуде из темного стекла.
	
\subsubsection*{Смесь без азота}
	
\paragraph*{}В питательный раствор азот входит в виде нитратов Ca(NO${_3}$)${_2}$ и KNO$_3$. Для того чтобы сохранить концентрацию ионов K+ и Ca2+ в данном растворе, нитраты калия и кальция заменяются на KCl и CaSO${_4}$ \textperiodcentered 2H${_2}O$ соответственно.
	
	\paragraph*{Расчет массы \hypertarget{potashyum_mass}{KCl}} которую нужно внести в питательную смесь вместо KNO$_3$, производится исходя из того что 1 моль KNO$_3$\footnote{молярная масса KNO$_3$ составляет 101 г/моль.} содержит 40 г калия. Массу калия, содержащегося в 510 мг KNO$_3$ можно  рассчитать по пропорции
	
	\begin{equation}
		\frac{110 g_{KNO_3} - 40 g_{K}}{0,51 g_{KNO_3} - x g_{K}}
	\end{equation}	 

\paragraph*{}Отсюда	

\begin{equation}
		x = \frac{40 * 0,51}{110} = 0,19
	\end{equation}
	
\paragraph*{}Определив количество калия, содержащиеся в 510 мг KNO$_3$ можно определить, какое количество KCl необходимо для того чтобы в растворе содержалось 0,19 г калия. Масса KCl так же рассчитывается по пропорции:

	\begin{equation}
		\frac{75 g_{KCl} - 39 g_{K}}{x g_{KCl} - 0,19 g_{K}}
	\end{equation}
	
\paragraph*{}Отсюда:

\begin{equation}
		x = \frac{75*0,19}{39} = 0,37
	\end{equation}
	
\paragraph*{}Таким образом, для  того чтобы сохранить в питательной смеси нужное количество калия, вместо 0,51 г KNO$_3$ в смесь необходимо внести 0,37 г KCl

\paragraph*{Расчет массы KCl и CaSO${_4}$ \textperiodcentered 2H${_2}O$} которую необходимо внести вместо Ca(NO${_3}$)${_2}$ проводят по аналогичной схеме.

\subsubsection*{Смесь без фосфора}

\paragraph*{}Фосфор в питательной смеси содержится в виде гидрофосфта калия - KH${_2}$PO${_4}$. При исключении из смеси фосфора данная соль заменяется на KCl.

\paragraph*{Расчет массы KH${_2}$PO${_4}$} производят аналогично расчетам массы \hyperlink{potashyum_mass}{калия}.

\begin{equation}
	\frac{KH{_2}PO{_4} - K}{136 - 39}
\end{equation}

\begin{equation}
		x = \frac{39*0,136}{136} = 0,04
	\end{equation}

\begin{equation}
	\frac{KCl - K}{75 - 39}
\end{equation}

\begin{equation}
		x = \frac{74*0,04}{39} = 0,08
	\end{equation}
	
\paragraph*{}Таким образом, вместо 0,136 г KH${_2}$PO${_4}$ необходимо взять 0,08  KCl.

\subsubsection*{Смесь без калия}

\paragraph*{}Вместо гидрофосфата калия KH${_2}$PO${_4}$ в питательную смесь необходимо внести гидрофосфат натрия NaH${_2}$PO${_4}$ а вместо нитрата калия KNO${_3}$ - нитрата натрия - NaNO${_3}$.

\paragraph*{}Расчет массы данных веществ производится описанным \hyperlink{potashyum_mass}{выше} способом:

\begin{equation}
	\frac{KH{_2}PO{_4} - P}{136 - 31}
\end{equation}

\begin{equation}
		x = \frac{31*0,136}{136} = 0,031
	\end{equation}
	
\begin{equation}
	\frac{NaH{_2}PO{_4} - P}{138 - 31}
\end{equation}
	
\begin{equation}
		x = \frac{138*0,031}{31} = 0,138
	\end{equation}

\paragraph{}Следовательно, на 1 л смеси необходимо взять 0,138 г гидрофосфата натрия. Аналогичным образом рассчитывают массу NaNO${_3}$ необходимую для замены в смеси нитрата калия KNO${_3}$.

	\subsection*{Вопросы для самоконтроля}
	
	\begin{itemize}
		
		\item Какие элементы относятся к макро-, а какие к микроэлементам?
		\item Какие элементы являются реутилизируемыми а какие не реутилизируемыми? 
		\item По каким признакам можно отличить нехватку у растения рутилизированных элементов от нехватки нереутилизируемых?
		\item В чем заключается роль \hypertarget{nitrogen}{азота} в обмене веществ растения? Каковы признаки нехватки данного элемента для растения?
		\item В чем заключается роль в обмене веществ растительного организма \hypertarget{PS}{фосфора и серы}? Каковы признаки нехватки данных элементов для растения?
		\item В чем заключается роль в обмене веществ растительного организма \hypertarget{MnMg}{марганца и магния}? Каковы признаки нехватки данных элементов для растения?
	\end{itemize}