%\section*{\lbtitle Рост корней пшеницы в чистой соли и смеси солей}
%\addcontentsline{toc}{section}{Рост корней пшеницы в чистой соли и смеси солей}

%\subsection*{Теоретические положения}

%\paragraph*{}Корни растения поглощают из окружающей среды минеральные элементы в виде \hyperlink{ion}{ионов}. Часто ионы оказывают взаимное влияние на скорость поглощения, которое можно охарактеризовать как синергизм или антагонизм. Так ионы, имеющие одинаковый заряд, взаимно тормозят друг друга, и напротив ионы с противоположным зарядом ускоряют поступление друг дуга в растение.

%\paragraph{}\efbox[margin=10pt,backgroundcolor=yellow]{
%	\begin{minipage}{0.95 \textwidth}
%\paragraph*{}\textbf{Антагонизм} ионов это конкуренция между ионами одного заряда при поступлении в растение и как следствие уменьшение интенсивности их поглощения. Антагонистами являются ионы H$^+$, K$^+$, NH${_4}{^+}$, Ca$^{2+}$, Mg$^{2+}$, Cl$^-$, NO${_3}{^-}$, HCO${_3}{^-}$, SO${_4}{^2-}$, H$_2$PO${_4}{^-}$.

%\paragraph*{}\hyperlink{sinergism}{\textbf{Синергизм}} это явление, когда один ион способствует лучшему поглощению другого, например, Ca$^{2+}$ и K$^+$; Cl$^-$ и NO${_3}{^-}$.
%	\end{minipage}
%	}

%\paragraph*{}Подбирая различные концентрации отдельных ионов, можно составить такую их комбинацию, при которой данные растения будут развиваться лучше всего. Такой раствор называют \hyperlink{mixture}{уравновешенным} - то есть раствор нескольких солей, в котором не проявляется отрицательное действие на растение отдельных компонентов. 

\paragraph*{}\textbf{Цель работы}: Определить различия в характере роста пшеницы в растворе чистой соли и смеси солей.

\paragraph*{}\textbf{Оборудование}: Десятидневные проростки пшеницы. Конические колбы на 100 мл марля.

\paragraph*{}\textbf{Реактивы}: Растворы химически чистых солей - $KCl$, CaCl${_2}$, $NaCl$

	\subsection*{Ход работы}
	
\paragraph*{}Данная работа рассчитана на два занятия. В ходе первого из них ведётся закладка опыта, а в ходе второго — обработка полученных результатов.
	
	\subsubsection*{Первое занятие}
	
	\paragraph*{}Используя аналитические весы, сделайте навески и приготовьте растворы солей, согласно схемы опыта. Налейте приготовленные растворы в конические колбы объёмом 100 мл и закройте их горла марлевыми крышками. Высадите на каждую марлевую крышку одинаковое количество заранее приготовленных подростков пшеницы. Следите за тем, чтобы корни проростков были погружены в воду.
	
	\subsubsection*{Второе занятие}
	
\paragraph*{}Спустя две недели измерьте высоту проростков, число и длину корней и сделайте соответствующие выводы о \hyperlink{growth_question}{влиянии} на рост корней чистой соли и смеси солей. Результат наблюдений запишите в таблицу \ref{evolution_dependance} \cite{tretiakow_fzr}. 
	
\begin{table}
\label{evolution_dependance}
\caption{Развитие растения в зависимости от состава питательной смеси}
\begin{tabularx}{\linewidth}{|X|X|X|X|X|}
\hline	Вариант	&	Раствор	&	Объем раствора мл	&	Длинна надземной части см	&	Длинна корней см	\\
\hline	1	&	Полная смесь	&	 	&	 	&	 	\\
\hline	2	&	KCl	&	 	&	 	&	 	\\
\hline	3	&	NaCl	&	 	&	 	&	 	\\
\hline	4	&	CaCl$_{2}$	&	 	&	 	&	 	\\
\hline															
\end{tabularx}
\end{table}
	
	\subsection*{Вопросы для самоконтроля}

	\begin{itemize}
		\item Какая частица называется \hypertarget{ion}{ионом}? Вспомните, какие ионы играют наиболее важную роль в жизни растений?
		\item \hypertarget{mixture}{Какой} раствор называется уравновешенным?
		\item Какую роль играют используемые в данном опыте минеральные элементы в жизни растений?
		\item Каковы механизмы поступления минеральных элементов в растение?
		\item В чем состоит суть явлений \hypertarget{sinergism}{синергизма и антогонизма} ионов?
		\item В каком растворе наблюдался наибольший рост \hypertarget{growth_question}{проростков}, а в каком рост проростков был сильнее всего угнетен? Почему? 
	\end{itemize}
	