%\section*{\lbtitle Определение зон роста в органах растения}
%\addcontentsline{toc}{section}{Определение зон роста в органах растения}

%\subsection*{Теоретические положения}

%\paragraph*{}Рост стебля в высоту, как было сказано выше происходит за счет клеток меристем: апикальной — на кончике стебля и интеркаллярных, расположенных в узлах стебля. Именно в местах расположения меристем наблюдается наиболее интенсивных рост стебля. Таким образом величина прироста стебля по всей длине будет неодинаковой.Таким образом величина прироста стебля по всей длине будет неодинаковой.

\paragraph*{}\textbf{Цель работы}: Определить расположение зон наиболее интенсивного роста побега

\paragraph*{}\textbf{Оборудование}: Проростки гороха с корнями длинной 1,5-2 см, проростки подсолнечника, чёрная туш и перо или черный маркер, древесные опилки, препаровальные иглы, миллиметровая бумага.

\subsection*{Ход работы}

\paragraph*{}Данная работа рассчитана на два занятия. В ходе первого из них ведётся закладка опыта, а в ходе второго — обработка полученных результатов.

	\subsubsection*{Первое занятие}
	
	\paragraph*{}Прорастите во влажных опилках семена гороха (5 штук). Для обеспечения строго вертикального роста корней, в опилках стеклянной палочкой проделайте вертикальные углубления глубиной в несколько сантиметров. 
	
	\paragraph*{}На предварительно подсушенный фильтровальной бумагой корень гороха нанесите метки, расстояние между которыми составляет 3 мм. Метки должны быть тонкими и хорошо заметными. 
	
	\paragraph*{}\warningsign С помощью туши и пера метки получаются более тонкими, а измерения более точными. Однако при нанесении меток тушью соблюдайте осторожность так как можете случайно поранить пером стебель или корень проростка.
	
	\paragraph*{}Для определения зон роста побега, нанесите метки на расстоянии 3 мм на побег проростка подсолнечника. Всего 10 меток, начиная от верхушки. Затем поместите проростки с нанесёнными метками тёмное место\footnote{В условиях недостатка света побег начнет интенсивно расти, вытягиваться и, вследствие этого рост будет выражен более отчетливо.}.
	
	\subsubsection*{Второе занятие}
	
	\paragraph*{}Измерьте расстояние между метками на корнях гороха и побегах подсолнечника, на основание данных измерения нескольких растений вычислите среднесуточный прирост корня и стебля на разных участках. Запишите результаты опыта в \ref{growth_form}.
	
	\paragraph*{}По результатам измерения постройте график роста корня, где по оси абсцисс откладывается номер отрезка, а по оси ординат - прирост (рисунок \ref{growth_graf}). 
	
\begin{table}[h!]
\centering
\label{growth_form}
\caption{Форма записи результатов}
\begin{tabular}{|c|c|c|c|c|c|c|c|c|c|c|c|c|c|c|c|}
\hline & \multicolumn{15}{|c|}{Зона прироста мм}  \\ \cline{2-16}
 Номер проростка & 1 & 2 & 3 & 4 & 5 & 6 & 7 & 8 & 9 & 10 & 11 & 12 & 13 & 14 & 15 \\
\hline 1 & & & & & & & & & & & & & & & \\
\hline 2 & & & & & & & & & & & & & & & \\
\hline 3 & & & & & & & & & & & & & & & \\
\hline 4 & & & & & & & & & & & & & & & \\
\hline 5 & & & & & & & & & & & & & & & \\
\hline


\end{tabular}

\end{table}

\begin{figure}[h!]
\centering
\label{growth_graf}
\caption{Форма для построения графика}
\begin{tikzpicture}
	%Raster zeichnen
	\draw [color=gray!50]  [step=5mm] (0,0) grid (14,7);
	% Achsen zeichnen
	\draw[->,thick] (0,0) -- (15,0) node[right] {$x$};
	\draw[->,thick] (0,0) -- (0,8) node[above] {$y$};
	% Achsen beschriften
	\draw (1,-.2) -- (1,0) node[below=4pt] {$\scriptstyle 1$};
	\draw (2,-.2) -- (2,0) node[below=4pt] {$\scriptstyle 2$};
	\draw (3,-.2) -- (3,0) node[below=4pt] {$\scriptstyle 3$};
	\draw (4,-.2) -- (4,0) node[below=4pt] {$\scriptstyle 4$};
	\draw (5,-.2) -- (5,0) node[below=4pt] {$\scriptstyle 5$};
	\draw (6,-.2) -- (6,0) node[below=4pt] {$\scriptstyle 6$};
	\draw (7,-.2) -- (7,0) node[below=4pt] {$\scriptstyle 7$};	
	\draw (8,-.2) -- (8,0) node[below=4pt] {$\scriptstyle 8$};
	\draw (9,-.2) -- (9,0) node[below=4pt] {$\scriptstyle 9$};
	\draw (10,-.2) -- (10,0) node[below=4pt] {$\scriptstyle 10$};
	\draw (11,-.2) -- (11,0) node[below=4pt] {$\scriptstyle 11$};
	\draw (12,-.2) -- (12,0) node[below=4pt] {$\scriptstyle 12$};
	\draw (13,-.2) -- (13,0) node[below=4pt] {$\scriptstyle 13$};
	\draw (14,-.2) -- (14,0) node[below=4pt] {$\scriptstyle 14$};
	\foreach \y in {0,0,0,1,2,3,4,5,6,7}
	\draw (-.1,\y) -- (.1,\y) node[left=4pt] {$\scriptstyle\y$};
	\node[label={[label distance=-4.0cm,text depth=-1ex,rotate=90]left:Прирост (мм)}] at (-1,.10) {};
	\node[label={[label distance=6cm,text depth=-8ex]right:Номер отрезка}] at (0,-0.1) {};
\end{tikzpicture}
\end{figure}
	
\paragraph*{}На основании наблюдений, \textbf{сделайте вывод} о том, где расположены зоны наиболее интенсивного роста растения. С расположением каких образовательных тканей они связаны.

	\subsection*{Вопросы для самоконтроля}

	\begin{itemize}
		\item \hypertarget{where_meristems_plased}{Где} расположены основные образовательные ткани растения?
		\item Какие \hypertarget{chem_regulators}{гормоны} регулируют рост растения, в чем особенности регуляции роста этими гормонами?
		\item Как растительные гормоны влияют на этапы \hypertarget{chem_ontogenesis}{онтогенеза} растительного организма?
		\item Что такое тропизмы? Какие основные тропизмы проявляет организм растения?
		\item Что такое клеточный цикл? Какие события происходят на каждом из этапов клеточного цикла?
		\item На каком из этапов онтогенез растительная клетка способна к делению?
	\end{itemize}
