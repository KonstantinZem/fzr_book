%\section*{\lbtitle Определение потенциального осмотического давления клеточного сока путем плазмолиза}

%\addcontentsline{toc}{section}{Определение потенциального осмотического давления клеточного сока путем плазмолиза}

%\subsection*{Теоретические положения}

%\paragraph*{}Клеточный сок является сложным раствором разнообразных органических и неорганических соединений. Как и всякий раствор, клеточный сок характеризуется потенциальным осмотическим давлением, которое зависит от его концентрации, т.е. от числа частиц растворенного вещества, находящихся в этом растворе.

%\paragraph{}\efbox[margin=10pt,backgroundcolor=yellow]{
%	\begin{minipage}{0.95\textwidth}
%\textbf{Потенциальное осмотическое давление} (ПОД) - это показатель, который характеризует максимальную способность клетки всасывать воду. 
%	\end{minipage}
%	}

%\paragraph*{}Очевидно, что чем большим осмотическим давлением обладает клетка, тем выше ее способность к поглощению воды. По этой причине значение ПОД может быть использовано для оценки возможности произрастания растения на почвах с различной водоудерживающей силой. Так, наибольшим осмотическим давлением характеризуются клетки \hyperlink{xserofites}{растений-ксерофитов}, которые вынуждены извлекать воду из очень сухих почв и удерживать ее в своих клетках. 

%\paragraph*{}Величина \hypertarget{p_osm_other_plants}{ПОД} постоянно колеблется и различается как у разных видов растений, так и у растений одного вида, в разные периоды их жизненного цикла.  Например, низкое осмотическое давление – около 0,1 МПа наблюдается у водных растений. Осмотическое давление, равное 20,0 МПа, обнаружено у галофита \textit{Atriplex confertifolia}. У большинства растений средней полосы осмотическое давление колеблется от 0,5 до 3 МПа \cite{fzr_jakushina}. Повышение осмотического давления клеточного сока во время засухи служит критерием степени обезвоженности клетки.

\begin{footnotesize}

\paragraph*{}\textbf{Цель работы}: экспериментально определить величину потенциального осмотического давления клетки;

\paragraph*{}\textbf{Оборудование}: Луковица с пигментированными чешуями, микроскоп, предметные и покровные стекла, бритвы, бюксы, градуированные пипетки;

\paragraph*{}\textbf{Реактивы}: растворы сахарозы 0,1 \gls{mol} и KNO$_3$, 0,1 \gls{mol};

\end{footnotesize}

\subsection*{Ход работы}

\subsubsection*{Наблюдение плазмолиза в растворах различной концентрации}

\paragraph*{}В бюксах необходимо приготовить по 10 мл растворов сахарозы и KNO$_3$ различной концентрации. Данные растворы готовятся путем разбавления дистиллированной водой исходного 1 М раствора сахарозы согласно таблице \ref{pot_osm_press_work}.

\begin{table}[!h]

\label{pot_osm_press_work}
\caption{Рабочая таблица для расчета потенциального осмотического давления}
\begin{tabular}{|p{2cm}|p{2cm}|p{1.5cm}|p{1.5cm}|p{1.5cm}|p{1.5cm}|p{1.5cm}|p{1.5cm}|}


\hline Концентра\-ция раствора, \- \gls{mol}/л & 1 моль раствора сахарозы, мл & воды мл & время погружения & время наблюдения & степень плазмолиза & изотони\-ческая концентра\-ция,\- \gls{mol}/л & осмоти\-ческое давление, к\gls{pascal} \\
%\hline \rotatebox{90}{Концентрация раствора, \- моль/л} & 1 моль раствора сахарозы, мл & воды мл & время погружения & время наблюдения & \rotatebox{90}{степень плазмолиза} & \rotatebox{90}{изотоническая концентрация,\- моль/л} & \rotatebox{90}{осмотическое давление, кПа} \\
\hline 0,7 & 7 & 3 &  &  &  & &  \\
\hline 0,6 & 6 & 4 &  &  &  & &  \\
\hline 0,5 & 5 & 5 &  &  &  & &  \\
\hline 0,4 & 4 & 6 &  &  &  & &  \\
\hline 0,3 & 3 & 7 &  &  &  & &  \\
\hline 0,2 & 2 & 8 &  &  &  & &  \\
\hline 0,1 & 1 & 9 &  &  &  & &  \\
\hline

\end{tabular}

\end{table}

\paragraph*{}Сделайте тонкий срез эпидермиса с выпуклой стороны пигментированной чешуи луковицы.

\paragraph*{}Затем, с интервалом времени в 3 минуты, поместите полученные срезы чешуй (по 2-3) в каждый из бюксов с приготовленными растворами, начиная с того бюкса, в котором концентрация раствора самая высокая (0,7 \gls{mol}). 
Через 30 минут извлеките срезы из первого бюкса и исследуйте их под микроскопом. Срезы из остальных бюксов извлекаются, через каждые 3 минуты и так же исследуются под микроскопом. 

\paragraph*{\warningsign}Срезы исследуются в капле того раствора, из которого их извлекли.

\paragraph*{}В каждом из исследуемых под микроскопом срезов определите степень выраженности плазмолиза. На основе этих наблюдений -- \textit{\hypertarget{c_isitonik}{изотоническую концентрацию}} клеточного сока, которая рассчитывается как среднее арифметическое между концентрацией раствора, в котором ещё не наблюдается плазмолиза и концентрации раствора, в котором плазмолиз уже начался.

\paragraph*{}Результаты опыта запишите в таблицу \ref{pot_osm_press_work}.

\paragraph*{}Потенциальное осмотическое давление цитоплазмы клетки рассчитывается по формуле \ref{p_osmotick_formula}:

\begin{equation}
\label{p_osmotick_formula}
	P = RTci
\end{equation}

\paragraph*{}Где \textit{P} - потенциальное осмотическое давление, \textit{R} - универсальная газовая постоянная Больцмана, равная 8,3 Дж/\gls{mol}*К, \textit{T} -- температура в Кельвинах, \textit{c} - изотоническая концентрация, найденная \hyperlink{c_isitonik}{ранее} опытным путем, \textit{i} - изотонический коэффициент Вант-Гоффа, который показывает степень ионизации раствора и определяется по формуле \ref{i_isotonik}:

\begin{equation}
	\label{i_isotonik}
	i = 1 + \alpha(n + 1)
\end{equation}

\paragraph*{}Где $\alpha$ - степень диссоциации раствора данной концентрации, n - число ионов, на которое диссоциирует данное вещество. Например, для KNO$_3$ n = 2, так как данная соль диссоциирует с образованием двух ионов:

\begin{equation}
	KNO_3 \rightarrow K{^+} + NO{_3}{^-}
\end{equation}

\paragraph*{}Для сахарозы, которая не является электролитом и не диссоциирует, n = 1.

\paragraph*{}Ниже приводится степень диссоциации KNO$_3$ в зависимости от концентрации раствора

\begin{table}[h!]
\centering
\caption{Степень диссоциации KNO$_3$ в зависимости от концентрации раствора}
	\begin{tabularx}{\linewidth}{|p{5cm}|X|X|X|X|X|}
	
	\hline Концентрация (\gls{mol})     & 0,5  & 0,4  & 0,3  & 0,2  & 0,1 \\
	\hline Степень диссоциации & 0,71 & 0,74 & 0,76 & 0,79 & 0,83 \\
	\hline
	\end{tabularx}
	
\paragraph*{}Таблица приведена согласно \cite{vorob_2013}
\end{table}

\paragraph*{}\textbf{Сделайте вывод}, в котором сопоставьте определенную вами в опыте величину осмотического давления клеток кожицы лука с величиной осмотического давления клеток других \hyperlink{p_osm_other_plants}{растений} , которые упоминаются в теоретических положениях к данному опыту.

\subsection*{Вопросы для самоконтроля}

	\begin{itemize}
		\item Какие растения относятся к \hypertarget{xserofites}{ксерофитам}? Какими чертами организации они обладают?
		\item Что такое коллигативные свойства раствора. Перечислите эти свойства;
		\item Почему во время засухи осмотическое давление в клетках растения повышается?
		\item Что такое водный потенциал клетки, каковы его составляющие?
		\item Какое практическое значение имеет определение величины осмотического давления клеток растения?
 	
	\end{itemize}