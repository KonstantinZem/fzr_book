%\section*{\lbtitle Выявление защитного действия сахаров на протоплазму}
%\addcontentsline{toc}{section}{Выявление защитного действия сахаров на протоплазму}

%\subsection*{Теоретические положения}

%\paragraph*{}Под воздействием отрицательных температур в межклетниках образуются кристаллы льда, которые оттягивают на себя воду из близлежащих клеток. Таким образом клетки подвергаются обезвоживанию, при определенной степени которого происходит коагуляция белков цитоплазмы. Кроме того, кристаллы льда могут образовываться и внутри клеток, нанося им тем самым механические повреждения.


%\paragraph*{}Увеличение в цитоплазме клеток концентрации растворимых сахаров увеличивает водоудерживающую способность цитоплазмы клеток.

\begin{footnotesize}

\paragraph*{}\textbf{Цель работы}: Наблюдение защитного действия сахаров на цитоплазму;
\paragraph*{}\textbf{Оборудование}: Корнеплод свеклы, термометры, скальпели, бритвенные лезвия, пробирки, микроскопы, предметные стекла, карандаш по стеклу, фильтровальная бумага;

\paragraph*{}\textbf{Реактивы}: раствор сахарозы концентрацией 0,1 и 0,5 M, NaCl, лед;

\end{footnotesize}

\subsection*{Ход работы}

\paragraph*{}Из корнеплода свёклы лезвием вырежьте высечки размером 5x5x5 см, которые затем промывают водой. Приготовленные таким образом высечки поместите в три пробирки по 3 высечки в каждую.

\paragraph*{}Первую пробирку наполните водой, вторую – 0,1 м, а третью - 0,5 м раствором сахарозы. Предварительно подписанные пробирки поместите на 20 минут в охладительную смесь, состоящую из трёх частей льда и одной части NaCl.

\paragraph*{}По прошествии 20-минут разморозьте пробирки и отметьте, как изменилась окраска жидкости в пробирках. Затем извлеките высечки из пробирок и с помощью острого лезвия изготовьте из них тонкие срезы, которые поместите на предметное стекло в капле того же растворе, в котором они находились. Подсчитайте количество обесцветившихся клеток в поле зрения микроскопа. Результат запишите в таблицу \ref{plasmolis_table}.

\begin{table}[h!]
\centering
\label{plasmolis_table}
\caption{Влияние раствора сахаров на морозоустойчивость цитоплазмы}
	\begin{tabular}{|c|c|c|c|c|}
	
		\hline Условия &	\multicolumn{2}{c|}{Число клеток в поле зрения} & Окр/Неокр клетки & Вывод \\ \cline{2-5}
		 & Окрашеные & Неокрашеные & & \\
		\hline Вода & & & & \\
		\hline Сахароза 0,5 моль & & & & \\
		\hline Сахароза 0,1 моль & & & & \\
		\hline
	
	\end{tabular}
\end{table}
