%\section*{\lbtitle Выявление апикального доминирования у гороха}
%\addcontentsline{toc}{section}{Выявление апикального доминирования у гороха}

%\subsection*{Теоретические положения}

%\paragraph*{}Как известно, у многих растений верхушка подавляет развитие боковых побегов и пробуждение спящих почек \cite{fzr_ermakov}. Это явление получило название апикальное доминирование. Удаление или повреждение верхушечной почки снимает апикальное доминирование, в следствии чего спящие почки пробуждаются а рост боковых побегов усиливается.

%\paragraph*{}Горох это растение, у которого апикальное доминирование выражено довольно сильно. Это можно обнаружить, если удалить верхушку главного побега. Рост боковых побегов начинается уже через несколько дней после удаления верхушки.

\paragraph*{}\textbf{Цель работы}: Выявить явление апикального доминирования на примере растений гороха.
\paragraph*{}\textbf{Оборудование}: Сосуды с молодыми растениями гороха, бритвы, линейки.

\subsection*{Ход работы}

	\subsubsection*{Первое занятие}
	
\paragraph*{}Среди имеющихся молодых растений, срежьте у двух  лезвием верхушку, а одно растение оставьте нетронутым в качестве контроля. Сосуды с растениями необходимо поместить в теплицу или климатостат.
	
	\subsubsection*{Второе занятие}
	
	\paragraph*{}На следующие занятие сравните высоту побега, количество и размер боковых побегов у интактных (не поврежденных) и декапилированных (лишенных верхушки) растений гороха. Результаты сравнения занесите в отчет в виде таблицы \ref{growth_results_table} \cite{tretiakow_fzr}. 
	
\begin{table}
\label{growth_results_table}
\caption{Форма записи результатов опыта}
\begin{tabularx}{\linewidth}{|X|X|X|X|}
\hline	\multirow{1}{*}{Растение}	&	\multirow{1}{*}{Число боковых побегов}	&	\multicolumn{2}{|c|}{Длинна боковых побегов} \\ \cline{3-4}
								&		                                    &	Каждого                     &	Суммарная	\\ 
\hline	Интактное	&	 	&	 	&		\\
\hline	Декапитированное 1	&	 	&	 	&		\\
\hline	Декапитированное 2	&	 	&	 	&		\\
\hline	В среднем	&	 	&	 	&		\\
\hline								
\end{tabularx}
\end{table}
	
\subsection*{Вопросы для самоконтроля}

\begin{itemize}
	\item Чем рост отличается от развития?
	\item Какие растительные гормоны-стимуляторы роста вы знаете? Какие из них стимулируют рост путем растяжения клеток, а какие путем деления клеток?
	\item Какие растительные ткани отвечают за рост растения? В чем заключаются гистологические особенности строения этих тканей?
	\item Опираясь на результаты данной лабораторной работы, опишите, в чем заключается смысл такого агротехнического приема как пикировка?
	\item Перечислите и охарактеризуйте этапы развития растительного организма.
\end{itemize}
	
%\chapter{Приспособление и устойчивость растений}

%\paragraph*{}\efbox[margin=10pt,backgroundcolor=yellow]{
%	\begin{minipage}{0.95 \textwidth}Стресс это общая неспецифическая адаптационная реакция организма на действие любых неблагоприятных факторов, которые носят название стрессоры.
%	\end{minipage}
%	}

%\paragraph*{}В процессе развития стрессовой реакции растительный организм проходит  три стадии: 
%\begin{itemize}
%	\item первичная стрессовая реакция,
%	\item адаптация,
%	\item истощение.
%\end{itemize}

%\paragraph*{}Устойчивость растений к стрессору зависит и от фазы онтогенеза. Так, наиболее устойчивы растения, находящиеся в состоянии покоя а наиболее чувствительны растения в молодом возрасте.
 
%\paragraph*{}Ответ на стрессовое воздействие прослеживается на разных уровнях организации. так на уровне клетки наблюдается  

%\begin{itemize}
%	\item Повышение проницаемости мембран, деполяризация мембранного потенциала плазмалеммы.
%	\item Сдвиг рН цитоплазмы в кислую сторону.
%	\item Усиление поглощения кислорода, ускоренная трата АТФ, развитие свободнорадикальных процессов.
%	\item	Активация синтеза стрессовых белков.
%\end{itemize}

%\paragraph*{}Эти реакции направлены на защиту внутриклеточных структур и устранение неблагоприятных изменений в клетках. В невысоких дозах повторяющиеся стрессы приводят к закаливанию организма. Закаливание к одному стрессору способствует повышению устойчивости организма и другим повреждающим факторам.
