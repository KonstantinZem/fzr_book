\subsubsection*{Структурная и функциональная организация растительной клетки}

\begin{enumerate}

\item Матрикс \hyperlink{cell_wall}{клеточной стенки} составляют следующие вещества:
\begin{enumerate}[label=\emph{\alph*})]
	\item целлюлоза, гемицеллюлоза;  
	\item пектиновые вещества, целлюлоза;  
	\item крахмал, пектиновые вещества; 
	\item гемицеллюлоза, пектиновые вещества, белок.
\end{enumerate}
	 

\item Какие формы воды, внутри растительной клетки можно выделить?
\begin{enumerate}
	\item гравитационная и пленочная; 
	\item свободная и труднодоступная;  
	\item свободная и связанная;  
	\item связанная и легкодоступная.
\end{enumerate}

\item В каких типах \hyperlink{cell_plastids}{пластид} осуществляется процесс фотосинтеза?
\begin{enumerate}
	\item хлоропласты;  
	\item лейкопласты; 
	\item хромопласты; 
	\item митохондрии. 
\end{enumerate}

\item Что составляет парапласт растительной клетки?
\begin{enumerate}
	\item \hyperlink{cell_vakuol}{вакуоль}, \hyperlink{cell_wall}{клеточная стенка}; 
	\item макроскопические структуры;  
	\item ядро, цитоплазма; 
	\item неживые внутриклеточные включения.
\end{enumerate}

\item Какие из перечисленных веществ можно отнести к макроэргичеcким соединениям?
\begin{enumerate}
	\item \hyperlink{proteins}{белки};  
	\item \hyperlink{sect_lipids}{жиры};  
	\item аминокислоты;  
	\item \gls{atp}, УТФ, сахарофосфаты. 
\end{enumerate}

\item В каком из вариантов наиболее правильно и полно перечислены компоненты, входящие в состав молекулы \gls{atp} ?
\begin{enumerate}
	\item рибоза, три остатка фосфорной кислоты, аденин;  
	\item рибоза, два остатка фосфорной кислоты, аденин; 
	\item рибоза, два остатка фосфорной кислоты, урацил;  
	\item рибоза, три остатка фосфорной кислоты, урацил.
\end{enumerate}

\item Как называется вещество, с которым взаимодействует \hyperlink{enzimes}{фермент}, образуя комплекс? 
\begin{enumerate}
	\item субстрат; 
	\item изофермент; 
	\item кофермент; 
	\item простетическая группа.
\end{enumerate}

\item При увеличении количества \hyperlink{enzimes}{фермента} скоростью ферментативной реакции...
\begin{enumerate}
	\item уменьшается; 
	\item увеличивается; 
	\item остается неизменной;
	\item увеличивается, затем уменьшается.
\end{enumerate}

\item Как называется часть молекулы фермента, которая соединяется с субстратом?
\begin{enumerate}
	\item ингибитором;  
	\item аллостерическим центром;  
	\item активным центром; 
	\item активатором.
\end{enumerate}

\item Как называется химическая связь, с помощью которой соединяются между собой аминокислоты в молекуле белка?
\begin{enumerate}
	\item пептидной; 
	\item водородной; 
	\item ионной; 
	\item Ван-дер-Ваальса. 
\end{enumerate}

\item Какая группа липидов используется растительной клеткой как запасные питательные вещества?
\begin{enumerate}
	\item собственно жиры; 
	\item фосфолипиды; 
	\item гликолипиды; 
	\item воска. 
\end{enumerate}

\end{enumerate}

\subsubsection*{Водный обмен растений}

\begin{enumerate}
\item У каких органов растений интенсивность транспирации выше?
\begin{enumerate}
	\item лист; 
	\item стебель; 
	\item корень; 
	\item цветок.
\end{enumerate}


\item По какой причине большинство растений не могут расти на сильно засоленных почвах?
\begin{enumerate}
	\item из-за высокого осмотического потенциала почвенного раствора в связи с чем корни не могут поглощать воду
	\item из-за низкого осмотического потенциала почвенного раствора в связи с чем корни не могут поглощать воду
	\item из-за низкого содержания воды в почве
	\item из-за токсичного действия на растение раствора соли
\end{enumerate}

\item В каком случае наиболее правильно и полно названы структуры, входящие в состав устьец?  
\begin{enumerate}
	\item замыкающие клетки с хлоропластами, у которых стенки, удаленные от устьичной щели, тоньше и поэтому более эластичные; 
	\item замыкающие клетки с хромопластами, бобовидной формы, устьичная щель;  
	\item замыкающие клетки, передний и задний дворик, устьичная щель; 
	\item две бобовидные клетки с большим количеством митохондрий.   
\end{enumerate}

\item Что является нижним концевым двигателем водного тока у растений?
\begin{enumerate}
	\item транспирация; 
	\item гуттация; 
	\item корневое давление; 
	\item адгезия.
\end{enumerate}

\item Основной поглощающей зоной корня, которая направляет воду в русло дальнего транспорта,  является зона 
\begin{enumerate}
	\item корневого чехлика; 
	\item деления (меристемы); 
	\item растяжения; 
	\item корневых волосков.
\end{enumerate}

\item Количество сухого вещества, которое образуется в растении при испарении 1 кг транспирированной воды, называется...
\begin{enumerate}
	\item коэффициентом водопотребления; 
	\item продуктивностью транспирации;
	\item интенсивностью транспирации;  
	\item транспирационным коэффициентом.
\end{enumerate}

\item Количество воды, испаряемой растением с единицы листовой поверхности в единицу времени, называется...
\begin{enumerate}
	\item коэффициентом водопотребления; 
	\item продуктивностью транспирации;
	\item интенсивностью транспирации;  
	\item транспирационным коэффициентом.
\end{enumerate}

\item Из каких явлений слагается водный режим растений?
\begin{enumerate}
	\item испарение воды
	\item поступление воды
	\item передвижение и испарение воды
	\item поступление, передвижение и испарение воды
\end{enumerate}

\item В каком случае наиболее полно и точно дано определение сосущей силы?
\begin{enumerate}
	\item Сила с которой клетка поглощает воду
	\item Сила, равная разности между осмотическим и тургорным давлением в клетке.
	\item Сила, равная сумме осмотического  и тургорного давления в клетке.
	\item Сила, с которой вода давит на внутреннюю часть оболочки клетки
\end{enumerate}

\item Выберите вариант, в котором наиболее правильно дано определение транспирации?
\begin{enumerate}
	\item Физиологически пассивное испарение воды наземными частями растений
	\item Физиологически активное испарение воды наземными частями растений
	\item Поступление воды через корневую систему
	\item Движение воды по проводящей системе растения
\end{enumerate}

\item По каким тканям по растениям движется вода с растворенными минеральными веществами?
\begin{enumerate}
	\item По ксилеме      
	\item По флоеме
	\item По апопласту  
	\item По симпласту
\end{enumerate}

\item Устьица у лиственных пород преимущественно располагаются на...
\begin{enumerate}
	\item нижней стороне листа  
	\item верхней части листа
	\item верхней стороне            
	\item обоих сторонах листа равномерно
\end{enumerate}

\item От каких факторов зависит прежде всего интенсивность транспирации?
\begin{enumerate}
	\item От влажности воздуха    
	\item От физиологического состояния растения
	\item От работы устьичного аппарата 
	\item От времени суток
\end{enumerate}

\item Как называется форма взаимодействия ионов в растворе, при которой суммарный эффект воздействия на растение много больше суммы каждого эффекта:
\begin{enumerate}
	\item антагонизм                      
	\item синергизм
	\item аддитивное действие      
	\item конкуренция
\end{enumerate}

\end{enumerate}

\subsubsection*{Фотосинтез}

\begin{enumerate}

\item В каком случае определение фотосинтеза дано наиболее полно
\begin{enumerate}
	\item Фотосинтез это процесс трансформации химической энергии органических соединений в энергию света; 
	\item Фотосинтез это процесс, при котором на свету в зеленых частях растений из углекислого газа и воды образуются органические вещества и высвобождается молекулярный кислород; 
	\item Фотосинтез это процесс выделения кислорода и поглощения углекислого газа; 
	\item Фотосинтез это процесс образования сложных органических веществ из простых при участии энергии света.
\end{enumerate}

\item Укажите фотосинтетические пигменты высших растений.
\begin{enumerate}
	\item антоцианы, хлорофиллы, каротиноиды; 
	\item каротины, ксантофиллы, хлорофиллы; 
	\item хлорофиллы,  антоцианы, каротины; 
	\item ксантофиллы, антоцианы, каротиноиды.
\end{enumerate}

\item Какие из пигментов являются вспомогательными при фотосинтезе?
\begin{enumerate}
	\item антоцианы; 
	\item каротиноиды; 
	\item хлорофилл а; 
	\item хлорофилл b.
\end{enumerate}

\item Назовите условия, необходимые для биосинтеза хлорофилла.
\begin{enumerate}
	\item наличие пластид, света, азота, магния, микроэлементов, воды, температура 15-25 \celsius; 
	\item наличие пластид, воды, углекислоты, температура 1-15 \celsius; 
	\item наличие углеводов, азота, магния, температура 15-25 \celsius; 
	\item наличие азота, микроэлементов, кислорода, температура 15-25 \celsius. 
\end{enumerate}

\item Акцептором $CO_{2}$ в цикле Кальвина является...
\begin{enumerate}
	\item фосфоенолпируват; 
	\item рибулезо-1,5-дифосфат; 
	\item рибозафосфат; 
	\item фосфоглицериновая кислота.
\end{enumerate}

\item У каких растений фотосинтез идет по пути С4?
\begin{enumerate}
	\item пшеница, ячмень, картофель, куриное просо; 
	\item кукуруза, просо, сорго, куриное просо, лебеда, сахарный тростник; 
	\item картофель, пшеница, ячмень, яблоня, одуванчик; 
	\item кукуруза, просо, сорго, ель, сосна, береза.
\end{enumerate}

\item Какую область спектра солнечного света принято считать за фотосинтетически активную радиацию (ФАР)? 
\begin{enumerate}
	\item 380-720 нм;  
	\item 290-380 нм; 
	\item 450-860 нм;  
	\item 720-4000 нм.
\end{enumerate}

\item В растительных клетках в отличие от животных происходит...
\begin{enumerate}
	\item хемосинтез        
	\item биосинтез белка
	\item фотосинтез       
	\item синтез липидов
\end{enumerate}

\item Какой элемент прежде всего необходим для образования хлорофилла?
\begin{enumerate}
	\item железо           
	\item магний
	\item марганец       
	\item медь
\end{enumerate}

\item Фотосинтетический аппарат растительной клетки локлизован в...
\begin{enumerate}
	\item клеточных мембранах          
	\item мембране хлоропластов
	\item строме хлоропластов            
	\item мембранах и строме хлоропластов
\end{enumerate}

\item От какой молекулы отщепляется кислород, выделяющийся в ходе фотосинтеза?
\begin{enumerate}
	\item $CO_{2}$                  
	\item $H_{2}O$
	\item $CO_{2}$ и $H_{2}O$       
	\item $C_{6}H_{12}O_{6}$
\end{enumerate}

\item Смысл эффекта Эмерсона заключается в том, что в процессе фотосинтеза участвует...
\begin{enumerate}
	\item одна фотосистема, поглощает свет длиной волны 700 нм
	\item две фотосистемы, поглощают свет с одной и той же длиной
волны
	\item две фотосистемы, поглощающие свет с разной длиной волны
\end{enumerate}

\item Какие пигменты содержатся в мембранах тилакоидов хлоропластов высших растений ?
\begin{enumerate}
	\item хлорофилл «а»          
	\item хлорофилл «б»
	\item каротин                      
	\item ксантофилл
\end{enumerate}

\item С химической точки зрения хлорофиллы являются...
\begin{enumerate}
	\item карбоновыми кислотами         
	\item ферментами
	\item эфирами                                     
	\item многоатомными спиртами
\end{enumerate}

\item У сине-зеленых водорослей световая фаза фотосинтеза
протекает в...
\begin{enumerate}
	\item хлоропластах                                
	\item хромопластах
	\item пиреноидах хроматофоров          
	\item тилакоидах
\end{enumerate}

\item Основными продуктами световой фазы фотосинтеза являются...
\begin{enumerate}
	\item углеводы                                                      
	\item \gls{atp}
	\item углеводы, \gls{atp} и \gls{nadfh2}                      
	\item \gls{atp} и \gls{nadfh2}
\end{enumerate}

\item Пигмент-белковый комплекс, включающий хлорофилл а максимумом поглощения 680 нм, феофитин, хиноны и другие компоненты, называется...
\begin{enumerate}
	\item PQ                 
	\item \gls{fs1}
	\item \gls{fs2}            
	\item ССК
\end{enumerate}

\item Какие вещества образуются в результате темновой фазы фотосинтеза?
\begin{enumerate}
	\item белки; 
	\item углеводы; 
	\item липиды; 
	\item нуклеиновые кислоты.
\end{enumerate}

\item Что такое компенсационная точка фотосинтеза?
\begin{enumerate}
	\item освещенность, при которой интенсивность фотосинтеза равна интенсивности  дыхания; 
	\item такое состояние, при котором количество образованного органического вещества больше, чем израсходованного при дыхании; 
	\item количество света, при котором начинается фотосинтез; 
	\item освещенность, при которой фотосинтез максимальный.
\end{enumerate}

\item Каким уравнением можно выразить процесс фотосинтеза?
\begin{enumerate}
	\item $6CO_{2} + 6H_{2}O   \rightarrow  C_{6}H_{12}O_{6} + 6O_{2}$ + энергия;
	\item $6CO_{2} + 6H_{2}O   \rightarrow  C_{6}H_{12}O_{6} + 6O_{2}$;
	\item $6CO_{2} + H_{2}O + свет \rightarrow   C_{6}H_{12}O_{6} + O_{2}$ ;
	\item $C_{6}H_{12}O_{6}  + 6O_{2}  → 6СО2  + 6H_{2}O$ + \gls{atp}
\end{enumerate}

\end{enumerate}


\subsubsection*{Минеральное питание растений}

\begin{enumerate}

\item Назовите ферменты,   которые в растении участвуют в восстановлении  нитратов до аммиака 
\begin{enumerate}
	\item нитрогеназа,  нитратредуктаза; 
	\item нитратредуктаза, нитритредуктаза; 
	\item нитритредуктаза, нитрогеназа; 
	\item нитрогеназа, аминотрансфераза.
\end{enumerate}

\item Что такое микориза и какова ее роль в жизни растений?
\begin{enumerate}
\item микроорганизмы на корнях растений и вокруг них, потребляющие и снижающие токсичность корневых выделений; 
\item сожительство грибов с корнями; увеличивается поглотительная способность и объем поглощаемых веществ из почвы; 
\item корневые выделения в прикорневой зоне; повышается растворимость минералов; 
\item прикорневая зона, богатая микроорганизмами, минерализующими органические вещества и растворяющими минералы почвы.
\end{enumerate}

\item Что такое уравновешенный раствор?
\begin{enumerate}
\item раствор, в котором нет токсического действия солей, количество и соотношение ионов в котором исключает их вредное влияние; 
\item раствор, в котором одна соль вызывает избыточное поглощение другой; 
\item почвенный раствор, если он имеет рН – 7; 
\item раствор, в котором добавление одних солей повышает эффективность использования других.
\end{enumerate}

\item Какой химический элемент относится к органогенным?
\begin{enumerate}
	\item Pb        
	\item Cl
	\item O          
	\item S
\end{enumerate}

\item Какой химический элемент не относится к макроэлементам?
\begin{enumerate}
	\item Fe        
	\item P
	\item S         
	\item K
\end{enumerate}

\item Какой химический элемент не относится к макроэлементам?
\begin{enumerate}
	\item P        
	\item S
	\item K        
	\item Ni
\end{enumerate}

\item Нитритредуктаза осуществляет катализ процесса:
\begin{enumerate}
	\item восстановление $NO^{-}_{3}$               
	\item восстановление молекулярного азота до аммония
	\item восстановление $NO^{-}_{2}$          
	\item аммонификация
\end{enumerate}

\item Какие реакции азотного обмена в растениях, требуют затраты НАДФН2:
\begin{enumerate}
	\item редукция нитратов                                          
	\item редукция нитритов
	\item первичное аминирование кетокислот          
	\item переаминирование
\end{enumerate}

\item Что такое микориза и какова ее роль в жизни растений?
\begin{enumerate}
	\item микроорганизмы на корнях растений и вокруг них, потребляющие и снижающие токсичность корневых выделений; 
	\item сожительство грибов с корнями; увеличивается поглотительная способность и объем поглощаемых веществ из почвы; 
	\item корневые выделения в прикорневой зоне; повышается растворимость минералов; 
	\item прикорневая зона, богатая микроорганизмами, минерализующими органические вещества и растворяющими минералы почвы.
\end{enumerate}

\item Что такое ризосфера и какова ее роль в питании растений?
\begin{enumerate}
	\item микроорганизмы на корнях растений и вокруг них, потребляющие и снижающие токсичность корневых выделений; 
	\item сожительство грибов с корнями; увеличивается поглотительная способность и объем поглощаемых веществ из почвы; 
	\item корневые выделения в прикорневой зоне; повышается растворимость минералов; 
	\item прикорневая зона, богатая микроорганизмами, минерализующими органические вещества и растворяющими минералы почвы.
\end{enumerate}

\item В каком процессе в биологическом круговороте азота, участвуют бактерии рода ризобиум:
\begin{enumerate}
	\item симбиотическая азотфиксация      
	\item несимбиотическая азотфиксация
	\item аммонификация                               
	\item нитрификация
\end{enumerate}

\item Какой химический элемент, участвует в образовании макроэргических связей:
\begin{enumerate}
	\item N           
	\item P
	\item S            
	\item K
\end{enumerate}

\item Какой химический элемент, участвует в образовании макроэргических связей:
\begin{enumerate}
	\item P            
	\item S
	\item K           
	\item Mg
\end{enumerate}

\item К какой группе элементов следует отнести азот?
\begin{enumerate}
	\item к макроэлементам;              
	\item к микроэлементам; 
	\item к ультрамикроэлементам; 
	\item к органогенам.
\end{enumerate}

\item В какое важнейшее органическое соединение в растениях, не входит азот:
\begin{enumerate}
	\item хлорофиллы       
	\item белки
	\item \gls{atp}                   
	\item \gls{pva}
\end{enumerate}

\item Нормальный биосинтез хлорофилла невозможен при голодании растений по:
\begin{enumerate}
	\item Са          
	\item S
	\item K           
	\item Mg
\end{enumerate}

\item К какой группе элементов следует отнести бор и медь?
\begin{enumerate}
	\item к макроэлементам;             
	\item к микроэлементам; 
	\item к ультрамикроэлементам; 
	\item к органогенам.
\end{enumerate}

\item Элемент, входящий в состав каталитических центров многих ферментов из класса оксидоредуктаз (цитохромы, каталазы, пероксидазы и др.):
\begin{enumerate}
	\item Fe             
	\item Са
	\item Сu           
	\item K
\end{enumerate}

\item Какой минеральный элемент растительной клетки, не входит в состав зольных:
\begin{enumerate}
	\item Р           
	\item N
	\item S           
	\item K
\end{enumerate}

\item Голодание по какому элементу характеризуется точечным хлорозом листьев, когда жилки остаются зелеными, а участки тканей, расположенные между жилками, бледнеют?
\begin{enumerate}
	\item S            
	\item K
	\item Mg       
	\item Ca
\end{enumerate}

\item Какой химический элемент входящий в состав растительной клетки относится к микроэлементам?
\begin{enumerate}
	\item P          
	\item S
	\item Mg       
	\item Cu
\end{enumerate}

\item Какой химический элемент входящий в состав растительной клетки относится к микроэлементам?
\begin{enumerate}
	\item P          
	\item S
	\item K          
	\item Ni
\end{enumerate}

\item Голодание по какому элементу характеризуется общим хлорозом листьев, когда листья принимают характерную желтую окраску?
\begin{enumerate}
	\item S          
	\item K
	\item Fe        
	\item Ca
\end{enumerate}

\item Какой химический элемент входящий в состав растительной клетки не относится к макроэлементам?
\begin{enumerate}
	\item P         
	\item S
	\item K        
	\item Mn
\end{enumerate}

\item Тесный контактный обмен между ризодермой и частицами почвы обеспечивается...
\begin{enumerate}
	\item переходом ионов в почвенный раствор
	\item прилипанием частиц почвы к корневым волоскам при выделении корневыми воосками слизи
	\item адсорбцией почвенных частиц клетками ризодермы
	\item отсутствием у ризодермы кутикулы
\end{enumerate}

\item Процесс в биологическом круговороте азота, в котором участвуют бактерии рода азотобактер носит название...
\begin{enumerate}
	\item симбиотическая азотфиксация    
	\item несимбиотическая азотфиксация
	\item аммонификация                            
	\item нитрификация
\end{enumerate}

\item Какие соединения не синтезируется в корнях растений?
\begin{enumerate}
	\item цитокинины         
	\item аминокислоты
	\item пигменты              
	\item азотистые основания.
\end{enumerate}

\item Какое соединение обеспечивает транспорт кислорода к бактероидам при симбиотической азотфиксации?
\begin{enumerate}
	\item леггемоглобин               
	\item молибден
	\item оксигеназа                     
	\item цитохромоксидаза
\end{enumerate}

\end{enumerate}

\subsubsection*{Рост и развитие растений}

\begin{enumerate}

\item  Что такое глубокий (органический) покой семян:
\begin{enumerate}
	\item  Покой связанный с особенностями внутреннего развития зародыша.
	\item  Покой, связанный с недостатком влаги
	\item  Покой связанный с неблагоприятной температурой
	\item  Покой связанный с особенностями внутреннего развития зародыша и неблагоприятными факторами
\end{enumerate}

\item Что такое общий, или физиологический, возраст:
\begin{enumerate}
	\item определяется от момента его заложения до момента исследования
	\item определяется календарным возрастом данного органа 
	\item определяется возрастом материнского организма в целом к моменту его заложения.
	\item определяется календарным возрастом данного органа и возрастом материнского организма в целом к моменту его заложения
\end{enumerate}

\item К каким анатомическим структурам приурочен апикальный рост побега
\begin{enumerate}
	\item К верхушечным меристемам        
	\item К латеральным меристемам
	\item К камбию                                        
	\item К феллогену
\end{enumerate}

\item Генеративные органы это:
\begin{enumerate}
	\item органы размножения                  
	\item все органы кроме органов размножения
	\item органы полового размножения  
	\item органы не полового размножения
\end{enumerate}

\item Фотопериодизм это:
\begin{enumerate}
	\item Реакция растений на соотношение продолжительности дня и ночи
	\item Реакция только на длину светового дня
	\item Реакция только на длину ночи
	\item Реакция растений на длину светового дня и соотношение продолжительности дня и ночи
\end{enumerate}

\item С помощью каких пигментов растения воспринимают длинну светового дня?
\begin{enumerate}
	\item листом с помощью пигмента фитохрома   
	\item листом с помощью пигмента хлорофила
	\item листом с помощью пигментов каратиноидов   
	\item листом с помощью пигментов антоцианов
\end{enumerate}

\item Какой тип роста органов характерен для стеблей и корней?
\begin{enumerate}
	\item интеркалярный;  
	\item апикальный; 
	\item базальный;  
	\item латеральный.
\end{enumerate}

\item Для каких органов растений характерен интеркалярный тип роста?
\begin{enumerate}
	\item кукурузы, картофеля; 
	\item соломины злаковых культур; 
	\item стеблей двудольных; 
	\item листьев двудольных.
\end{enumerate}

\item Для какой части растений характерен отрицательный геотропизм?\
\begin{enumerate}
	\item для надземной части растений; 
	\item для листьев двудольных; 
	\item для корневой системы; 
	\item для стеблей злаковых.
\end{enumerate}

\item Как называются ростовые движения растений, обусловленные диффузными факторами внешней среды?
\begin{enumerate}
	\item корреляция; 
	\item тропизмы; 
	\item настии; 
	\item таксисы.
\end{enumerate}

\item Индивидуальное развитие растительного организма, начинающееся с образования зиготы и заканчивающееся биологической смертью, называется
\begin{enumerate}
	\item онтогенез; 
	\item органогенез; 
	\item эмбриогенез; 
	\item метаморфоз.
\end{enumerate}

\item Стимуляция цветения растений при действии пониженных температур называется 
\begin{enumerate}
	\item термонастии; 
	\item фотопериодизм; 
	\item яровизация; 
	\item фотопериодическая индукция.
\end{enumerate}

\item Фаза дифференциации клетки характеризуется...
\begin{enumerate}
	\item образованием вторичной клеточной оболочки, усилением специализации клеток; 
	\item активным нарастанием новых тканей и органов растений, усилением интенсивности дыхания, повышением концентрации фитогормонов; 
	\item усилением гидролитических процессов, распадом сложных органических соединений на более простые, повышением концентрации клеточного сока за счет осмотически активных веществ; 
	\item усилением клеточного деления, образованием макроэргических соединений.
\end{enumerate}

\item В каком случае наиболее правильно названы типы покоя, характерные для растений?
\begin{enumerate}
	\item относительный, абсолютный; 
	\item глубокий, временный; 
	\item глубокий, вынужденный; 
	\item абсолютный, глубокий.
\end{enumerate}

\item Полярность растительных клеток и органов растения это...
\begin{enumerate}
	\item это ростовое движение побега и клеток
	\item взаимное влияние частей, органов растений, тканей на характер их роста и развития; 
	\item физиологическая неравноценность противоположных полюсов клетки, органа и целого растения; 
	\item восстановление утраченных частей растения.
\end{enumerate}

\end{enumerate}

\subsubsection*{Дыхание}

\begin{enumerate}

\item В каких структурах происходит процесс дыхания у растений?
\begin{enumerate}
	\item в специальных органах                     
	\item во всех живых клетках
	\item только в клетках с хлоропластами   
	\item только в молодых клетках
\end{enumerate}

\item К какому классу относятся ферменты, которые участвуют в процессе переноса электронов и водорода при дыхании?
\begin{enumerate}
	\item оксидоредуктазы; 
	\item трансферазы; 
	\item лиазы; 
	\item изомеразы.
\end{enumerate}

\item В каких органоидах осуществляется глиоксилатный цикл дыхания?
\begin{enumerate}
	\item глиоксисомы; 
	\item пероксисомы; 
	\item рибосомы; 
	\item митохондрии.
\end{enumerate}

\item Процесс восстановления кислорода из воды и окисления субстрата до $CO_{2}$ в ходе внутриклеточного дыхания...
\begin{enumerate}
	\item разделены во времени протекания
	\item разделены в пространстве
	\item разделены во времени и пространстве
	\item объединены во времени протекания и в пространстве
\end{enumerate}

\item Макроэргические связи в молекуле \gls{atp} образованы...
\begin{enumerate}
	\item остатками фосфорной кислоты        
	\item аминогруппой в аденине
	\item группами ОН в рибозе                       
	\item связью аденина с рибозой
\end{enumerate}

\item Какое вещество, является промежуточным продуктом окисления субстрата и при дыхании, и при брожении?
\begin{enumerate}
	\item \gls{acetylCoensimA}           
	\item глюкоза
	\item \gls{pva}                        
	\item молочная кислота
\end{enumerate}

\item Величина дыхательного коэффициента растительной клетки свидетельствует о...
\begin{enumerate}
	\item эффективности (\gls{kpd}) дыхания
	\item интенсивности дыхания
	\item природе основного субстрата дыхания (типе дыхательного
обмена)
	\item пути окисления глюкозы
\end{enumerate}

\item Гликолиз является процессом...
\begin{enumerate}
	\item циклическим, аэробным            
	\item линейным, аэробным
	\item циклическим, анаэробным        
	\item линейным, анаэробным
\end{enumerate}

\item Чему равен дыхательный коэффициент для органических кислот?
\begin{enumerate}
	\item больше 1; 
	\item равен 1; 
	\item меньше 1; 
	\item равен 0.
\end{enumerate}

\item Что такое окислительное фосфорилирование?
\begin{enumerate}
	\item процесс образования молекул \gls{atp} при дыхании; 
	\item процесс, при котором затрачивается энергия \gls{atp} при синтезе органических веществ; 
	\item третий этап анаэробной фазы дыхания; 
	\item первый этап аэробной фазы дыхания.
\end{enumerate}

\end{enumerate}
