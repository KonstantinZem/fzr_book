\paragraph*{}Этот электронно-методический комплекс не является полноценным учебником и не предназначен для полноценного самостоятельного изучения физиологии растений. Его нужно рассматривать как своеобразную <<дорожную карту>>, посредством которой студент, сможет узнать что его ждет в процессе изучения курса <<Физиология и биохимия растений>>, либо, после очередного занятия повторить пройденный материал, проверить, все ли сказал преподаватель, все ли было занесено в конспект. Повторюсь, здесь приведен конспект лекций, однако понимая, что конспект может прочитать только его создатель (и то не всегда), я постарался немного расширить лекции примерами и словами связками.

\paragraph*{}В данном комплексе приведены конспекты лекций, описание лабораторных работ, словарь основных терминов, а так же примеры тестовых заданий и экзаменационных вопросов.

\paragraph*{}Отвечая на вопросы, приведенные в конце каждого раздела, студент сможет проконтролировать успешность освоения дисциплины.

\paragraph*{}\note{В таких рамках помещен текст, необходимый для логической связки между обязательными записями, либо дополнительный материал, приведенный для лучшего понимания темы}

\paragraph*{}\remember{В таких рамках помещены важные для запоминания вещи}

\paragraph*{}Автор будет очень благодарен, если вы укажите ему на найденные неточности и неясности и сообщите о них автору по адресу \href{mailto:konstantinz@bk.ru}{konstantinz@bk.ru}