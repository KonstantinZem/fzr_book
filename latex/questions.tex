\begin{enumerate}
	\item Структурные компоненты клетки
	\item Химический состав растительной клетки: Неорганические компонент
	\item Химический состав растительной клетки: Углеводы
	\item Химический состав растительной клетки: Жиры
	\item Химический состав растительной клетки: Липиды
	\item Химический состав растительной клетки: Белки
	\item Химический состав растительной клетки: Ферменты, витамины, макроэргические соединения 
	\item Классификация и номенклатура ферментов
	\item Классификация и биологическая роль витаминов
	\item Содержание и формы воды в почве
	\item Растительная клетка как осмотическая система
	\item Транспорт воды в растении
	\item Транспирация и ее роль для растений
	\item. Фотосинтез. Световая фаза
	\item Структура фотосинтетического аппарата растений
	\item Фотосинтетическое фосфорилирование
	\item Фотолиз воды
	\item Фотосинтез. Темновая фаза. Цикл  Кальвина
	\item Интенсивность фотосинтеза и методы ее определения
	\item Влияние густоты стояния растений, особенностей расположения листьев, и других факторов на эффективность фотосинтеза
	\item Дыхание растений. Современные представления о химизме дыхания. Дыхание как универсальный окислительный процесс
	\item Аэробное дыхание. Цикл Кребса. Пентозо-фосфатный путь
	\item Дыхательная электрон-транспортная цепь
	\item Анаэробное дыхание (Брожение). Уксуснокислое брожение, cпиртовое брожение
	\item Регулирование дыхания при хранении сельскохозяйственных культур
	\item Влияние условий хранения на интенсивность дыхания семян и плодов.
	\item Пути организации оптимальных условий для хранения плодов и семян
	\item Корень как орган поглощения элементов минерального питания
	\item Азотное питание растений
	\item Почва, как источник минеральных элементов для растений. Физиологические основы применения удобрений
	\item Особенности роста органов растений. Влияние экологических факторов на рост
	\item Развитие растений. Понятие об онтогенезе
	\item Физиология цветения. Физиология покоя семян. Экзогенный и эндогенный покой
	\item Способы прекращения и продления покоя семян. Физиологические основы хранения семян
	\item Физиология и биохимия формирования качества урожая сельскохозяйственных культур
	\item Приспособление и устойчивость растений. Устойчивость к температуре. Холодостойкость, Морозоустойчивость, Жароустойчивость
	\item Приспособление и устойчивость растений. Устойчивость к недостатку и избытку влаги. Засухоустойчивость
	\item Солеустойчивость, газоустойчивость растений. Действие на растения радиации и пестицидов
	\item Устойчивость растений к действию биотических факторов
	\item Устойчивость к инфекционным заболеваниям
	\item Аллелопатические взаимодействия культурных растений и сорняков.
	\item Растение как саморегулирующаяся система
	\item Системы саморегуляции и интеграции у растений
	\item Взаимодействие растений в ценозах
\end{enumerate}