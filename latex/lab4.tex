%\section*{\lbtitle Наблюдение за движением устьичных клеток под микроскопом} 

%\subsection*{Теоретические положения}

%\paragraph*{}Через устьица приходит испарение воды и газообмен между растением и атмосферой. Каждое устьице состоит из замыкательных клеток бобовидной (у двудольных) или гантелевидной (у однодольных) растений формы. Работа устьиц обеспечивается неравномерным утолщением их клеточных стенок, за счет чего, замыкающие клетки устьиц в состоянии тургора выгибаются, открывая, таким образом, устьичную щель. За счет работы устьиц растение может активно регулировать уровень испарения воды с поверхности своих листьев.

\begin{footnotesize}

\paragraph*{}\textbf{Цель работы}: Наблюдение за работой устьичного аппарата под микроскопом;

\paragraph*{}\textbf{Оборудование}: Листья традесканции, лезвия, препаровальные иглы, предметные и покровные стекла, фильтровальная бумага, микроскоп;

\paragraph*{}\textbf{Реактивы}: Раствор сахарозы 1 М; Раствор глицерина 5\%, вода;

\end{footnotesize}

\subsection*{Ход работы}

\subsubsection*{Наблюдение работы устьичного аппарата в растворе сахарозы}

\paragraph*{}На предметное стекло, в каплю воды, поместите срез эпидермиса нижней стороны листа. Рассмотрите под микроскопом устьица, зарисуйте в рабочей тетради строение устьица и отметить на рисунке утолщения внутренней стенки замыкательных клеток устьиц. 

\paragraph*{}Затем, с помощью фильтровальной бумаги замените воду на предметном стекле на раствор сахарозы и наблюдайте за изменениями устьица. В рабочей тетради нужно зарисовать устьице в закрытом состоянии. Снова замените воду на предметном стекле на раствор сахарозы и наблюдайте за закрытием устьица.

\subsubsection*{Наблюдение работы устьичного аппарата в растворе глицерина}

\paragraph*{}Поместить срез эпидермиса листа в раствор глицерина и наблюдать плазмолиз в замыкательных клетках устьиц. В следствии плазмолиза устьичные клетки закрываются, но через некоторое время открываются вновь. Это происходит из за того, что глицерин проникает в цитоплазму устьичных клеток, вызывая их деплазмолиз.

\paragraph*{}Замените глицерин на предметном стекле на воду. Для этого с одной стороны капните на предметное стекло каплю воды, а с другой оттяните глицерин с помощью фильтровальной бумаги. Устьица откроются еще шире, чем это было в начале опыта, так как после проникновения в цитоплазму замыкательных клеток глицерина, осмотическое давление в этих клетках начинает повышаться.

\paragraph*{}Результаты наблюдений запишите в тетрадь.

\paragraph*{}\textbf{Сделайте вывод} как проведенные вами наблюдения согласуются с известными вами сведениями о механизме работы устьиц.

\subsection*{Вопросы для самоконтроля}

\begin{itemize}
	\item Какое приспособительное значение имеет расположение устьиц на нижней стороне листа?
	\item Какое значение для работы устьица имеет неравномерное утолщение стенки замыкательной клетки устьица?
	\item В какое время суток устьица в растении находятся в раскрытом состоянии. Почему?
	\item Почему осмотическое давление клеточного сока замыкательных клеток устьиц повышается на свету?
\end{itemize}

