\subsection{Пояснительная записка}

\subsubsection{Актуальность изучения дисциплины}
\paragraph*{}Физиология и биохимия растений - биологическая дисциплина, которая обеспечивает интеграцию всех биологических знаний на уровне целого растения и агрофитоценоза. Она является теоретической основой агрономических наук: растениеводства, плодоводства, овощеводства, агрохимии, защиты растений, селекции, хранения и переработки продукции растениеводства, биотехнологии, экологии. Знания, полученные в ходе изучения данного курса помогут агроному-агрохимику, плодоовощеводу, садоводу-декоратору с помощью агротехнических приемов, удобрений, пестицидов и физиологически активных веществ управлять ростом и развитием растений, влиять на качество производимой продукции.

\paragraph*{}Для успешного изучения физиологии и биохимии необходимо иметь прочные знания по ботанике, цитологии, генетике, микробиологии, физике, химии.

\subsubsection{Цель и задачи учебной дисциплины}

\paragraph*{}Цель дисциплины: 

\begin{enumerate}

\item изучение процессов, протекающих в растениях, как на молекулярном уровне, так и на уровне клеточных и субклеточных структур, тканей, органов, целого растения
\item изучение процессов, протекающих в агрофитоценозах: в посевах и насаждениях сельскохозяйственных растений. 

\end{enumerate}


\paragraph*{}Задачи дисциплины:

\begin{enumerate}

\item изучение физико-химической сущности и роли важнейших физиологических процессов: фотосинтеза, дыхания, водообмена, минерального питания, роста и развития, адаптации и устойчивости растений;
\item освоение практических приемов регулирования светового, теплового и водного режимов растений в посевах и насаждениях; 
\item получение навыков определения основных физиологических показателей; 
\item освоение методов количественного и качественного анализа растений и продукции растениеводства;
\item формирование целостного представления о физиолого-биохимических процессах, происходящих в растениях, о взаимосвязи растений в агрофитоценозах и их роли в экосистемах.

\end{enumerate}

\subsubsection{Требования к уровню освоения содержания учебной дисциплины} 

\paragraph*{}В результате изучения данной дисциплины студент должен закрепить и развить следующие академические (АК), социально-личностные (СЛК), профессиональные компетенции (ПК), предусмотренные в образовательном стандарте ОСВО 1-74 02 01-2013 .

\begin{enumerate}

\item АК-6. Владеть междисциплинарным подходом при решении проблем.
\item АК-3. Владеть исследовательскими навыками.
\item АК-4. Уметь работать самостоятельно.
\item АК-5. Быть способным вырабатывать новые идеи.
\item СЛК-1. Обладать качествами гражданственности.
\item СЛК-2. Быть способным к социальному взаимодействию.
\item СЛК-3. Обладать способностью к межличностным коммуникациям.

\end{enumerate}

\paragraph*{}В результате изучения дисциплины студент должен обладать следующими профессиональными компетенциями (ПК), предусмотренными образовательным стандартом ОСВО 1-74 02 01-2013:

\begin{enumerate}

\item ПК-2. Совершенствовать и оптимизировать действующие технологические схемы на базе системного подхода к анализу режимов и параметров операций и процессов.
\item ПК-18. Работать с научной, нормативно-справочной и специальной литературой, международной электронной системой.
\item ПК-20. Уметь работать с нормативной и юридической литературой и трудовым законодательством.

\end{enumerate}

\paragraph*{}Для приобретения профессиональных компетенций ПК-2, 18, 20 в результате изучения дисциплины студент должен:

\paragraph*{}\textbf{знать}: 

\begin{enumerate}

\item общие закономерности жизнедеятельности растений и их зависимость от условий среды;
\item химический состав растений, свойства и обмен основных химических компонентов клеток, их биологическую и энергетическую ценность;
\item физиолого-биохимические особенности формирования урожая сельскохозяйственных культур, влияние почвенно-климатических условий, орошения и удобрений на урожайность и качество продукции растениеводства;
\item механизмы устойчивости растений к холоду, морозу, засухе, токсичным газам, засолению, пестицидам, радиоактивному излучению, биотическим факторам;

\end{enumerate}

\paragraph*{}\textbf{уметь}: 

\begin{enumerate}

\item объяснять и прогнозировать ход физиолого-биохимических процессов в зависимости от условий среды;
\item управлять процессами жизнедеятельности растений с целью повышения урожайности и улучшения качества продукции растениеводства;
\item определять жизнеспособность растительных тканей при воздействии на них различных факторов;
\item оценивать экологическую безопасность продукции растениеводства; 

\end{enumerate}

\paragraph*{}\textbf{владеть}:

\begin{enumerate}

\item навыками физиолого-биохимических исследований.
\item приемами управления ростом и развитием растений для повышения урожайности и качества продукции растениеводства;
\item способами повышения устойчивости растений к неблагоприятным условиям среды.

\end{enumerate}

\subsubsection{Структура содержания учебной дисциплины}

\paragraph*{}Содержание дисциплины представлено в виде тем, которые характеризуются относительно самостоятельными укрупненными дидактическими единицами содержания обучения.

\paragraph*{}Учебная дисциплина рассчитана на 217 академических часов (5 зач. ед.), в том числе 102 часа — аудиторных, из них: 34 часа лекционных, 68 часов лабораторных. Форма контроля — зачет, экзамен.

\subsubsection{Методы (технологии) обучения}

\paragraph*{}Преподавание должно осуществляться с использованием современных методов, приемов, технических и других учебных средств, а также форм обучения, направленных на оптимизацию и интенсификацию процесса обучения.

\paragraph*{}Основными методами (технологиями) обучения, отвечающими целям изучения дисциплины, являются:

\begin{enumerate}

\item элементы проблемного обучения (проблемная ситуация, проблемное изложение, обучающе-исследовательский метод);
\item элементы учебно-исследовательского подхода, реализация творческого подхода, реализуемые в самостоятельной работе;
\item элементы коммуникативного обучения (индивидуализация обучения, новизна 
в обучении).

\end{enumerate}

\subsubsection{Организация самостоятельной работы студентов}

\paragraph*{}При изучении дисциплины используются следующие формы самостоятельной работы:

\begin{enumerate}

\item изучение теоретического материала не только в учебниках и учебных пособиях, указанных в библиографических списках, но и ознакомление с публикациями в периодических изданиях; 
\item выполнение индивидуальных заданий методического характера;
\item подготовка внеаудиторных вопросов к темам курса с консультациями преподавателя;
\item работа по заполнению форм учебной документации.

\end{enumerate}

\subsubsection{Диагностика компетенций студентов}

\paragraph*{}Оценка учебных достижений студентов осуществляется на зачете и по результатам промежуточных учебных достижений, используя критерии, утвержденные Министерством образования Республики Беларусь. 

\paragraph*{}Для оценки учебных достижений студентов используется следующий диагностический инструментарий:

\begin{enumerate}

\item проведение тестирования (АК-1; ПК-1, ПК-2, ПК-5);
\item защита выполненных самостоятельных заданий (АК-1–АК-3);
\item защита выполненных на лабораторных занятиях индивидуальных заданий (АК-1; ПК-1–ПК-3; ПК-5);
\item сдача экзамена по дисциплине (АК-1–АК-3; СЛК-1; ПК-1; ПК-3; ПК-5). 

\end{enumerate}

\subsection{Содержание учебного материала}

\subsubsection{Ведение}

\paragraph*{}Предмет и задачи физиологии и биохимии растении, ее место в системе биологических дисциплин. Физиология и биохимия как фундаментальная основа агрономических наук и биотехнологии.

\paragraph*{}Этапы развития физиологии и биохимии растений как науки, вклад в неё отечественных и зарубежных ученых. Основные направления современной физиологии и биохимии растении. Методы и уровни последовании физиологии и биохимии растений.

\subsubsection{Структурная и функциональная организация растительной клетки}


\paragraph*{Тема 1.1 Структурная организация растительной клетки.}

\paragraph*{}Клетка как структурная и функциональная единица растительного организма. Основные черты жизнедеятельности растительной клетки. Принцип компартментализации - основа жизнедеятельности клетки. Гомеостаз, его значение для функционирования биологических систем.

\paragraph*{}Строение, свойства и функции структурных компонентов клетки: клеточной стенки, мембран, ядра, цитоплазмы, пластид, митохондрий, аппарата Гольджи эндоплазматической сети, лизосом, сфероеом. рибосом, пероксисом. вакуолей, микротрубочек. микрофиламентов. Взаимосвязь клеток в растительных тканях, апопласт, симпласт. Влияние токсичных веществ и радиации на свойства и функции мембран, цитоплазмы и органоидов клетки.

\paragraph*{}Проницаемость мембран и цитоплазмы. Механизмы транспорта веществ через мембраны, пассивный и активный транспорт. Закономерности диффузии, осмоса, электрофореза. Электрогенные и электронейтральные ионные насосы. Мембранный потенциал. Потенциалы покоя и действия.

\paragraph*{Тема 1.2 Химический состав растительной клетки}

\paragraph*{}Химический состав растительной клетки. Основные химические компоненты клетки, их классификация по происхождению и выполняемым функциям.

\paragraph*{}Углеводы: содержание, свойства и роль в растениях моно-, олпго- и полисахаридов.

\paragraph*{}Липиды, их содержание и роль в растениях.

\paragraph*{}Жиры: кислотный состав, свойства и функции.

\paragraph*{}Липоиды: строение, свойства и функции фосфоглицеридов, гликолипидов, восков, стероидов.

\paragraph*{}Белки, их состав, структура, свойства и функции. Классификация белков по строению и растворимости. Аминокислотный и фракционный состав белков. Протеиногенные, свободные и незаменимые аминокислоты. Биологическая питательная ценность белков.

\paragraph*{}Нуклеиновые кислоты, их строение, виды и функции.

\paragraph*{}Ферменты, их природа, строение, свойства и биологическая роль. Одно- и двухкомпонентные ферменты. Природа коферментов и простетических групп. Активные и аллостерические центры ферментов. Механизм действия ферментов.

\paragraph*{}Кинетика ферментативных реакций. Влияние температуры, кислотности среды, активаторов, ингибиторов и других факторов на скорость ферментативных реакций. Локализация ферментов и регуляция ферментативной активности в клетке.

\paragraph*{}Классификация и номенклатура ферментов. Участие отдельных представителей классов в окислительно-восстановительных, гидролитических, синтетических и других реакциях. Участие ферментных систем в иммобилизации и детоксикации пестицидов и их метаболитов. Изоферменты и их роль в повышении адаптивных свойств растений и устойчивости к болезням. Применение растворимых и иммобилизованных ферментов в сельском хозяйстве, промышленности, науке и технике.

\paragraph*{}Витамины, их классификация, свойства и биологическая роль. Каталитическая и регуляторная функции витаминов.

\paragraph*{}Макроэргические соединения клетки, их классификация и роль. АТФ и пути ее образования.

\paragraph*{}Раздражимость клетки. Действие инфекции и токсикантов на структуру и функции органоидов клетки. 

\subsubsection{Водный обмен растений}

\paragraph*{Тема 2.1 Поступление и транспорт воды в растение.}

\paragraph*{}Содержание, состояние, формы и роль воды в растениях. Термодинамические основы водообмена: активность и химический потенциал воды. Водный потенциал клетки и его компоненты. Явление набухания. Растительная клетка как осмотическая система. Осмотические явления в клетке - тургор, плазмолиз, циторриз, их значение в водообмене и жизнедеятельности растений.

\paragraph*{}Почва как среда водообеспечения растений, виды почвенной влаги и их доступность растениям. Корневая система как орган поглощения воды, поглотительная способность различных зон корня. Корневое давление, его природа, размеры, зависимость от условий среды. Плач и гуттация.

\paragraph*{}Транспорт воды в растении. Основные двигатели водного тока. Скорость передвижения воды по растению.

\paragraph*{}Транспирация, ее размеры и роль. Транспирация устьичная, кутикулярная и перидермальная. Физиология устьичных движений. Фотоактивное, гидроактивное и гидропассивное движения устьиц. Показатели транспирации: интенсивность транспирации, транспирационный коэффициент, продуктивность транспирации, относительная транспирация. Зависимость транспирации от условий среды, суточный ход. Способы снижения уровня транспирации. Антитранспиранты.

\paragraph*{Тема 2.2 Водный баланс и водный дифицит растения}

\paragraph*{}Водный баланс и водный дефицит растений. Влияние недостатка воды на растения. Виды завядания. Влияние избытка воды в почве на растения.

\paragraph*{}Водный режим в посевах сельскохозяйственных культур. Эвапотранспирация, эвапорация, коэффициент водопотреблення. Пути повышения эффективности использования воды растениями.

\paragraph*{}Физиологические основы орошения сельскохозяйственных культур. Физиологические показатели, применяемые для у становления необходимости полива. Использование параметров водообеспечепности растений при программировании урожаев.

\subsubsection{Фотосинтез}

\paragraph*{Тема 3.1 Структурная организация фотосинтетического аппарата. Световая фаза фотосинтеза}

\paragraph*{}Фотосинтез – основа энергетики биосферы и продукционного процесса растений. Современные представления о механизме фотосинтеза.

\paragraph*{}Структурная организация фотосинтетического аппарата. Лист как орган фотосинтеза. Радиационный баланс листа. Хлоропласта, их состав, строение, онтогенез. Фотосинтетические пигменты: хлорофиллы, каротиноиды, фикобилины. Их строение, химические и оптические свойства. Фотосинтетическая активная радиация (ФАР). Организация и функционирование пигментных систем. Светособирающий комплекс, реакционный центр, фотосистема I и II, электронно-транспортная цепь фотосинтеза. Миграция энергии в процессе фотосинтеза.

\paragraph*{}Световая фаза фотосинтеза. Циклическое и нециклическое фотосинтетическое фосфорилирование. Фотолиз воды.

\paragraph*{Тема 3.2 Темновая фаза фотосинтеза. Методы определения интенсивности фотосинтеза}

\paragraph*{}Темновая фаза фотосинтеза. Метаболизм углерода при фотосинтезе у С3-растений (цикл Кальвина) и С4-растений (цикл Хэтча-Слэка). Фотосинтез по типу толстянковых растении (CAM-метаболизм). Фотодыхание и его роль.

\paragraph*{}Интенсивность фотосинтеза и методы ее определения. Эндогенные механизмы регуляции фотосинтеза. Зависимость фотосинтеза от факторов внешней среды. Компенсационные точки. Фотосинтез как саморегулируемый процесс.
\paragraph*{}Посевы и насаждения как фотосинтезирующие системы. Параметры оценки фотосинтетической активности фитоценозов: индекс листовой поверхности, фотосинтетический потенциал, чистая продуктивность фотосинтеза, коэффициент полезного действия (КПД) фотосинтеза. Радиационный режим и структура посева. Параметры оптимального посева.

\paragraph*{Тема 3.3 Фотосинтез и урожай}

\paragraph*{}Фотосинтез и урожай. Урожай биологический и хозяйственный.
\paragraph*{}Влияние густоты стояния растении и структуры посева, особенностей расположения листьев в пространстве, удобрении, орошения на энергетическую эффективность агрофитоценозов. Пути повышения КПД ФАР в посевах. Использование показателей фотосинтеза при программировании урожая.

\paragraph*{}Светокультура сельскохозяйственных растений. Источники облучения. Влияние искусственного облучения на растения. Выращивание растении при искусственном освещении.

\subsubsection{Дыхание}

\paragraph*{Тема 4.1 Дыхание как универсальный окислительный процесс в живой природе.}

\paragraph*{}Дыхание как универсальный окислительный процесс в живой природе. Значение дыхания. Типы окислительно-восстановительных реакций и ферментные системы дыхания. Субстраты дыхания и энергетическая эффективность их использования. Дыхательный коэффициент.

\paragraph*{Тема 4.2 Современные представления о химизме дыхания}

\paragraph*{}Химизм дыхания: Пути дыхательного обмена. Гликолитический путь. Анаэробная фаза дыхания. Аэробная фаза дыхания. Цикл ди- и трикарбоновых кислот (цикл Кребса). Окислительный пентозофосфатный путь. Дыхательная электрон-транспортная цепь. Окислительное фосфорилирование. Энергетическая эффективность различных путей окис­ления. Химизм и энергетика анаэробного дыхания (брожения).

\paragraph*{Тема 4.3 Роль дыхания в процессах биосинтеза}

\paragraph*{}Связь дыхания и фотосинтеза. Использование энергии дыхания на рост и поддержание гомеостаза. Интенсивность дыхания, методы ее учета, зависимость от внутренних и внешних факторов. Дыхание больного растения.
Регулирование дыхания при хранении сельскохозяйственной продукции.

\subsubsection{Минеральное питание растений}

\paragraph*{Тема 5.1 Корень как орган поглощения элементов минерального питания}

\paragraph*{}История развития учения о корневом питании растений. Содержание и физиологическая роль в растениях макро- и микроэлементов, их соединения. Принципы диагностики дефицита питательных элементов. Физиологические нарушения при недостатке необходимых элементов питания.

\paragraph*{}Корень как орган поглощения элементов минерального питания. Поглощение минеральных веществ клетками корня. Поступление веществ в свободное пространство корня, перенос ионов и молекул через мембрану. Особенности поглощения клетками корня радионуклидов, пути снижения их поступления в растения. Ионный .транспорт в растении - внутриклеточный, ближний и дальний. Поглощение ионов клетками листа. Отток ионов из листьев. Перераспределение и реутилизация веществ в растении. Регулирование растением скорости поглощения ионов.

\paragraph*{}Взаимодействие ионов, аддитивность, антагонизм и синергизм. Физиологическая реакция солей. Физиологически уравновешенные растворы.

\paragraph*{Тема 5.2 Азотное питание растении}

\paragraph*{}Особенности усвоения нитратного и аммонийного азота. Ассимиляция нитратного азота. Причины накопления избыточного количества нитратов в растениях и пути их снижения в продукции растениеводства. Ассимиляция аммиака. Особенности азотного питания бобовых культур.
\paragraph*{}Корневая система как орган синтеза и выделения веществ. Корневые выделения.

\paragraph*{Тема 5.3 Физиологические основы применения удобрений}

\paragraph*{}Почва как источник минеральных элементов для растений. Влияние ризосферной микрофлоры на поглощение веществ. Микотрофный способ питания растений. Взаимодействие между растениями.
Некорневое питание растений.
Особенности питания растений в беспочвенной культуре (гидро- и аэропоника).

\subsubsection{Обмен и транспорт органических веществ в растениях}

\paragraph*{}Общие закономерности обмена веществ в растениях. Взаимосвязь обмена веществ и обмена энергии. Анаболизм и катаболизм на различных этапах онтогенеза растений. Биосинтез и взаимное превращение углеводов, липидов, аминокислот, белков, веществ вторичного происхождения при прорастании и созревании семян, плодов, клубней, луковиц и корнеплодов. Факторы, влияющие на направленность обменов веществ в растениях.

\paragraph*{}Транспорт органических веществ по флоэме. Состав флоэмного сока и скорость его перемещения. Транспортные формы органических веществ и донорно-акцепторные отношения в растении, аттрагирующие зоны. Научные гипотезы, объясняющие транспорт веществ по флоэме. Регуляция транспорта веществ в растениях. Способы управления транспортом веществ с целью повышения урожайности сельскохозяйственных культур и улучшения качества продукции.

\subsubsection{Рост и развитие растений}

\paragraph*{Тема 7.1 Рост растений}

\paragraph*{}Понятие об онтогенезе, росте и развитии растений. Периодизация онтогенеза. Клеточные основы роста и развития. Фитогормоны как факторы, регулирующие рост и развитие растений, их классификация, химическая природа, локализация и транспорт. Особенности действия фитогормонов на рост тканей и органов, формирование семян и плодов, морфогенез растений. Взаимодействие фитогормонов. Использование фитогормонов и физиологически активных веществ в сельскохозяйственной практике.

\paragraph*{}Локализация зон роста у высших растений. Особенности роста органов растений. Зависимость роста от внутренних факторов. Ростовые явления: периодичность и ритмичность роста, закон большого периода роста, ростовые корреляции, полярность. Методы измерения скорости роста.

\paragraph*{}Влияние экологических факторов на рост. Свет как фактор, регулирующий рост растений. Фитохромная система растений. Влияние температуры, влажности почвы и воздуха, аэрации, минерального питания, химических средств защиты растений и ксенабиотиков на рост растений. Необратимые нарушения роста. Карликовость и гигантизм. Ритмы физиологических процессов. Движения растений. Тропизмы и настии, их виды и значение.

\paragraph*{Тема 7.2 Развитие растений}

\paragraph*{}Теории развития растений. Морфологические, физиологические и биохимические признаки общих возрастных изменений у растений. Влияние внешних условий на развитие растений. Яровизация и термопериодизм. Фотопериодизм.

\paragraph*{}Физиология старения растений. Циклическое старение и омоложение растений и их органов в онтогенезе. Понятие о росте целостного растения. Управление генеративным развитием и старением растений. Особенности роста растений в фитоценозе. Регуляция роста и онтогенеза.

\paragraph*{}Физиология цветения, опыления и оплодотворения растений. Формирование семян как эмбриональный период онтогенеза растений. Накопление и превращение веществ при формировании семян. Превращение веществ при формировании сочных плодов. Приёмы нормирования плодоношения и ускорения созревания плодов и овощей. Влияние внутренних и внешних факторов на созревание и качество семян и плодов.

\subsubsection{Физиология и биохимия формирования качества урожая}

\paragraph*{}Физиология покоя семян. Экзогенный и эндогенный покой. Послеуборочное дозревание семян. Способы прекращения и продления покоя. Процессы, протекающие при прорастании семян. Физиологические основы хранения семян, плодов, овощей, кормов.

\paragraph*{}Физиология и биохимия формирования качества урожая сельскохозяйственных культур

\paragraph*{}Роль генетических и внешних факторов в интенсификации синтеза запасных веществ в различных органах растений. Основные физиолого-биохимичеекие процессы, происходящие при формировании продуктивных органов зерновых, зернобобовых, масличных, овощных, плодово-ягодных культур, картофеля, корнеплодов, волокнистых растений, кормовых трав. Влияние почвенно-климатических факторов, удобрений, орошения и агротехники на химический состав растений и качество продукции растениеводства.

\paragraph*{}Физиолого-биохимические аспекты улучшения экологической чистоты растительной продукции.

\subsubsection{Приспособление и устойчивость растений}

\paragraph*{Тема 9.1 Общие представления о стрессе. Устойчивость растений к высокой и низкой температуре}

\paragraph*{}Границы приспособления и устойчивости растений. Понятие о стрессе и стрессорах. Защитно-приспособительные реакции растений на действие повреждающих факторов. Обратимые и необратимые повреждения растении, их тканей и органов. Изменения физико-химических и функциональных свойств растительных клеток и тканей при повреждениях, процессы адаптации.
Холодостойкость. Физиолого-биохимические изменения у теплолюбивых растений при пониженных положительных температурах. Приспособление растений к низким положительным температурам. Способы повышения холодостойкости растении. Заморозки, зашита растений от заморозков.

\paragraph*{}Морозоустойчивость. Условия и причины вымерзания растении. Закаливание растений, его фазы. Обратимость процессов закаливания. Способы повышения морозоустойчивости. Методы изучения морозоустойчивости растений.

\paragraph*{}Зимостойкость как устойчивость к комплексу неблагоприятных факторов перезимовки. Выпреваине. Вымокание. Гибель под ледяном коркой. Выпирание. Повреждение от зимней засухи. Способы повышения зимостойкости растений. Меры предупреждения гибели озимых хлебов. Методы определения жизнеспособности сельскохозяйственных культур в зимний и ранневесенний периоды.

\paragraph*{}Жароустойчивость растений. Изменения в обмене веществ, росте и рати тип растений при действии максимальных температур. Диагностика жароустойчнвости. Способы повышения жароустойчивости растении.

\paragraph*{Тема 9.2 Устойчивость растений к избытку и недостатку влаги}

\paragraph*{}Засухоустойчивость растений. Совместное действие недостатка влаги и высокой температуры на растение. Особенности водообмена у растений различных экологических групп. Диагностика жаро- и засухоустойчивости. Физиологические особенности засухоустойчивости сельскохозяйственных растений. Предпосевное повышение жаро-  и засухоустойчивости. Пути повышения засухоустойчивости культурных растений.

\paragraph*{}Влияние на растения избытка влаги (устойчивость к переувлажнению). Факторы устойчивости против затопления.

\paragraph*{}Полегание растений и его причины (устойчивость к полеганию). Способы предупреждения полегания.

\paragraph*{}Солеустойчнвость растений. Влияние засоленности на растения, механизмы толерантности. Типы галофитов. Солеустойчивость культурных растений. Диагностика солеустойчивости. Возможности повышения солеустойчивости.

\paragraph*{Тема 9.3 Устойчивость растений к другим абиотическим факторам}

\paragraph*{}Газоустойчивость растений. Виды токсичных газов, выделяемых промышленностью и транспортом, пути их поступления в растения. Действие вредных газообразных веществ на растения. Особенности газоустойчивых растений. Защита окружающей среды от токсичных газов.

\paragraph*{}Действие пестицидов на растения. Поглощение пестицидов растениями. Транспорт и метаболизм пестицидов. Остаточное количество свободных и связанных пестицидов в продукции растениеводства. Устойчивость растений к пестицидам.

\paragraph*{}Действие радиации на растение. Виды радиоактивного излучения и их действие на генетический аппарат, структурные компоненты клетки, физиологические процессы растений. Радиочувствительность различных органов растений, различных видов растений, ее изменчивость в онтогенезе.

\paragraph*{Тема 9.4 Устойчивость растений к биотическим факторам}

\paragraph*{}Устойчивость сельскохозяйственных растений к действию биотических факторов. Устойчивость растений к инфекционным заболеваниям. Аллелопатические взаимодействия в ценозе. Аллелопатическое взаимодействие культурных растений и сорняков. Возможности ослабления негативных аллелопатических эффектов в посевах сельскохозяйственных растений.

\subsubsection{Растение как саморегулирующаяся и саморазвиваюшаяся адаптивная система}

\paragraph*{}Растение как саморегулирующаяся и саморазвивающаяся адаптивная система. Системы регуляции и интеграции у растений. Механизмы регуляции физиологических процессов: внутриклеточные, межклеточные, организменные. Взаимодействие растений в ценозах. Использование физиологических методов и показателей в технологиях возделывания сельскохозяйственных культур, научных исследованиях, мониторинге окружающей среды.